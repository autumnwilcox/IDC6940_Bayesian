% Options for packages loaded elsewhere
% Options for packages loaded elsewhere
\PassOptionsToPackage{unicode}{hyperref}
\PassOptionsToPackage{hyphens}{url}
\PassOptionsToPackage{dvipsnames,svgnames,x11names}{xcolor}
%
\documentclass[
  letterpaper,
  DIV=11,
  numbers=noendperiod]{scrartcl}
\usepackage{xcolor}
\usepackage[margin=1in]{geometry}
\usepackage{amsmath,amssymb}
\setcounter{secnumdepth}{5}
\usepackage{iftex}
\ifPDFTeX
  \usepackage[T1]{fontenc}
  \usepackage[utf8]{inputenc}
  \usepackage{textcomp} % provide euro and other symbols
\else % if luatex or xetex
  \usepackage{unicode-math} % this also loads fontspec
  \defaultfontfeatures{Scale=MatchLowercase}
  \defaultfontfeatures[\rmfamily]{Ligatures=TeX,Scale=1}
\fi
\usepackage{lmodern}
\ifPDFTeX\else
  % xetex/luatex font selection
\fi
% Use upquote if available, for straight quotes in verbatim environments
\IfFileExists{upquote.sty}{\usepackage{upquote}}{}
\IfFileExists{microtype.sty}{% use microtype if available
  \usepackage[]{microtype}
  \UseMicrotypeSet[protrusion]{basicmath} % disable protrusion for tt fonts
}{}
\makeatletter
\@ifundefined{KOMAClassName}{% if non-KOMA class
  \IfFileExists{parskip.sty}{%
    \usepackage{parskip}
  }{% else
    \setlength{\parindent}{0pt}
    \setlength{\parskip}{6pt plus 2pt minus 1pt}}
}{% if KOMA class
  \KOMAoptions{parskip=half}}
\makeatother
% Make \paragraph and \subparagraph free-standing
\makeatletter
\ifx\paragraph\undefined\else
  \let\oldparagraph\paragraph
  \renewcommand{\paragraph}{
    \@ifstar
      \xxxParagraphStar
      \xxxParagraphNoStar
  }
  \newcommand{\xxxParagraphStar}[1]{\oldparagraph*{#1}\mbox{}}
  \newcommand{\xxxParagraphNoStar}[1]{\oldparagraph{#1}\mbox{}}
\fi
\ifx\subparagraph\undefined\else
  \let\oldsubparagraph\subparagraph
  \renewcommand{\subparagraph}{
    \@ifstar
      \xxxSubParagraphStar
      \xxxSubParagraphNoStar
  }
  \newcommand{\xxxSubParagraphStar}[1]{\oldsubparagraph*{#1}\mbox{}}
  \newcommand{\xxxSubParagraphNoStar}[1]{\oldsubparagraph{#1}\mbox{}}
\fi
\makeatother

\usepackage{color}
\usepackage{fancyvrb}
\newcommand{\VerbBar}{|}
\newcommand{\VERB}{\Verb[commandchars=\\\{\}]}
\DefineVerbatimEnvironment{Highlighting}{Verbatim}{commandchars=\\\{\}}
% Add ',fontsize=\small' for more characters per line
\usepackage{framed}
\definecolor{shadecolor}{RGB}{241,243,245}
\newenvironment{Shaded}{\begin{snugshade}}{\end{snugshade}}
\newcommand{\AlertTok}[1]{\textcolor[rgb]{0.68,0.00,0.00}{#1}}
\newcommand{\AnnotationTok}[1]{\textcolor[rgb]{0.37,0.37,0.37}{#1}}
\newcommand{\AttributeTok}[1]{\textcolor[rgb]{0.40,0.45,0.13}{#1}}
\newcommand{\BaseNTok}[1]{\textcolor[rgb]{0.68,0.00,0.00}{#1}}
\newcommand{\BuiltInTok}[1]{\textcolor[rgb]{0.00,0.23,0.31}{#1}}
\newcommand{\CharTok}[1]{\textcolor[rgb]{0.13,0.47,0.30}{#1}}
\newcommand{\CommentTok}[1]{\textcolor[rgb]{0.37,0.37,0.37}{#1}}
\newcommand{\CommentVarTok}[1]{\textcolor[rgb]{0.37,0.37,0.37}{\textit{#1}}}
\newcommand{\ConstantTok}[1]{\textcolor[rgb]{0.56,0.35,0.01}{#1}}
\newcommand{\ControlFlowTok}[1]{\textcolor[rgb]{0.00,0.23,0.31}{\textbf{#1}}}
\newcommand{\DataTypeTok}[1]{\textcolor[rgb]{0.68,0.00,0.00}{#1}}
\newcommand{\DecValTok}[1]{\textcolor[rgb]{0.68,0.00,0.00}{#1}}
\newcommand{\DocumentationTok}[1]{\textcolor[rgb]{0.37,0.37,0.37}{\textit{#1}}}
\newcommand{\ErrorTok}[1]{\textcolor[rgb]{0.68,0.00,0.00}{#1}}
\newcommand{\ExtensionTok}[1]{\textcolor[rgb]{0.00,0.23,0.31}{#1}}
\newcommand{\FloatTok}[1]{\textcolor[rgb]{0.68,0.00,0.00}{#1}}
\newcommand{\FunctionTok}[1]{\textcolor[rgb]{0.28,0.35,0.67}{#1}}
\newcommand{\ImportTok}[1]{\textcolor[rgb]{0.00,0.46,0.62}{#1}}
\newcommand{\InformationTok}[1]{\textcolor[rgb]{0.37,0.37,0.37}{#1}}
\newcommand{\KeywordTok}[1]{\textcolor[rgb]{0.00,0.23,0.31}{\textbf{#1}}}
\newcommand{\NormalTok}[1]{\textcolor[rgb]{0.00,0.23,0.31}{#1}}
\newcommand{\OperatorTok}[1]{\textcolor[rgb]{0.37,0.37,0.37}{#1}}
\newcommand{\OtherTok}[1]{\textcolor[rgb]{0.00,0.23,0.31}{#1}}
\newcommand{\PreprocessorTok}[1]{\textcolor[rgb]{0.68,0.00,0.00}{#1}}
\newcommand{\RegionMarkerTok}[1]{\textcolor[rgb]{0.00,0.23,0.31}{#1}}
\newcommand{\SpecialCharTok}[1]{\textcolor[rgb]{0.37,0.37,0.37}{#1}}
\newcommand{\SpecialStringTok}[1]{\textcolor[rgb]{0.13,0.47,0.30}{#1}}
\newcommand{\StringTok}[1]{\textcolor[rgb]{0.13,0.47,0.30}{#1}}
\newcommand{\VariableTok}[1]{\textcolor[rgb]{0.07,0.07,0.07}{#1}}
\newcommand{\VerbatimStringTok}[1]{\textcolor[rgb]{0.13,0.47,0.30}{#1}}
\newcommand{\WarningTok}[1]{\textcolor[rgb]{0.37,0.37,0.37}{\textit{#1}}}

\usepackage{longtable,booktabs,array}
\usepackage{calc} % for calculating minipage widths
% Correct order of tables after \paragraph or \subparagraph
\usepackage{etoolbox}
\makeatletter
\patchcmd\longtable{\par}{\if@noskipsec\mbox{}\fi\par}{}{}
\makeatother
% Allow footnotes in longtable head/foot
\IfFileExists{footnotehyper.sty}{\usepackage{footnotehyper}}{\usepackage{footnote}}
\makesavenoteenv{longtable}
\usepackage{graphicx}
\makeatletter
\newsavebox\pandoc@box
\newcommand*\pandocbounded[1]{% scales image to fit in text height/width
  \sbox\pandoc@box{#1}%
  \Gscale@div\@tempa{\textheight}{\dimexpr\ht\pandoc@box+\dp\pandoc@box\relax}%
  \Gscale@div\@tempb{\linewidth}{\wd\pandoc@box}%
  \ifdim\@tempb\p@<\@tempa\p@\let\@tempa\@tempb\fi% select the smaller of both
  \ifdim\@tempa\p@<\p@\scalebox{\@tempa}{\usebox\pandoc@box}%
  \else\usebox{\pandoc@box}%
  \fi%
}
% Set default figure placement to htbp
\def\fps@figure{htbp}
\makeatother


% definitions for citeproc citations
\NewDocumentCommand\citeproctext{}{}
\NewDocumentCommand\citeproc{mm}{%
  \begingroup\def\citeproctext{#2}\cite{#1}\endgroup}
\makeatletter
 % allow citations to break across lines
 \let\@cite@ofmt\@firstofone
 % avoid brackets around text for \cite:
 \def\@biblabel#1{}
 \def\@cite#1#2{{#1\if@tempswa , #2\fi}}
\makeatother
\newlength{\cslhangindent}
\setlength{\cslhangindent}{1.5em}
\newlength{\csllabelwidth}
\setlength{\csllabelwidth}{3em}
\newenvironment{CSLReferences}[2] % #1 hanging-indent, #2 entry-spacing
 {\begin{list}{}{%
  \setlength{\itemindent}{0pt}
  \setlength{\leftmargin}{0pt}
  \setlength{\parsep}{0pt}
  % turn on hanging indent if param 1 is 1
  \ifodd #1
   \setlength{\leftmargin}{\cslhangindent}
   \setlength{\itemindent}{-1\cslhangindent}
  \fi
  % set entry spacing
  \setlength{\itemsep}{#2\baselineskip}}}
 {\end{list}}
\usepackage{calc}
\newcommand{\CSLBlock}[1]{\hfill\break\parbox[t]{\linewidth}{\strut\ignorespaces#1\strut}}
\newcommand{\CSLLeftMargin}[1]{\parbox[t]{\csllabelwidth}{\strut#1\strut}}
\newcommand{\CSLRightInline}[1]{\parbox[t]{\linewidth - \csllabelwidth}{\strut#1\strut}}
\newcommand{\CSLIndent}[1]{\hspace{\cslhangindent}#1}



\setlength{\emergencystretch}{3em} % prevent overfull lines

\providecommand{\tightlist}{%
  \setlength{\itemsep}{0pt}\setlength{\parskip}{0pt}}



 


\usepackage{booktabs}
\usepackage{longtable}
\usepackage{array}
\usepackage{multirow}
\usepackage{wrapfig}
\usepackage{float}
\usepackage{colortbl}
\usepackage{pdflscape}
\usepackage{tabu}
\usepackage{threeparttable}
\usepackage{threeparttablex}
\usepackage[normalem]{ulem}
\usepackage{makecell}
\usepackage{xcolor}
\KOMAoption{captions}{tableheading}
\makeatletter
\@ifpackageloaded{caption}{}{\usepackage{caption}}
\AtBeginDocument{%
\ifdefined\contentsname
  \renewcommand*\contentsname{Table of contents}
\else
  \newcommand\contentsname{Table of contents}
\fi
\ifdefined\listfigurename
  \renewcommand*\listfigurename{List of Figures}
\else
  \newcommand\listfigurename{List of Figures}
\fi
\ifdefined\listtablename
  \renewcommand*\listtablename{List of Tables}
\else
  \newcommand\listtablename{List of Tables}
\fi
\ifdefined\figurename
  \renewcommand*\figurename{Figure}
\else
  \newcommand\figurename{Figure}
\fi
\ifdefined\tablename
  \renewcommand*\tablename{Table}
\else
  \newcommand\tablename{Table}
\fi
}
\@ifpackageloaded{float}{}{\usepackage{float}}
\floatstyle{ruled}
\@ifundefined{c@chapter}{\newfloat{codelisting}{h}{lop}}{\newfloat{codelisting}{h}{lop}[chapter]}
\floatname{codelisting}{Listing}
\newcommand*\listoflistings{\listof{codelisting}{List of Listings}}
\makeatother
\makeatletter
\makeatother
\makeatletter
\@ifpackageloaded{caption}{}{\usepackage{caption}}
\@ifpackageloaded{subcaption}{}{\usepackage{subcaption}}
\makeatother
\usepackage{bookmark}
\IfFileExists{xurl.sty}{\usepackage{xurl}}{} % add URL line breaks if available
\urlstyle{same}
\hypersetup{
  pdftitle={Bayesian Logistic Regression for Predicting Diabetes Risk Using NHANES 2013--2014 Data},
  pdfauthor={Namita Mishra; Autumn Wilcox},
  colorlinks=true,
  linkcolor={blue},
  filecolor={Maroon},
  citecolor={Blue},
  urlcolor={Blue},
  pdfcreator={LaTeX via pandoc}}


\title{Bayesian Logistic Regression for Predicting Diabetes Risk Using
NHANES 2013--2014 Data}
\usepackage{etoolbox}
\makeatletter
\providecommand{\subtitle}[1]{% add subtitle to \maketitle
  \apptocmd{\@title}{\par {\large #1 \par}}{}{}
}
\makeatother
\subtitle{A Capstone Project on Bayesian Applications in Epidemiologic
Modeling}
\author{Namita Mishra \and Autumn Wilcox}
\date{2025-11-10}
\begin{document}
\maketitle

\renewcommand*\contentsname{Table of contents}
{
\hypersetup{linkcolor=}
\setcounter{tocdepth}{3}
\tableofcontents
}

Slides: \url{slides.html} (Edit \texttt{slides.qmd}.)

\section{Introduction}\label{introduction}

Diabetes mellitus (DM) remains a major public health challenge, and
identifying key risk factors---such as obesity, age, sex, and
race/ethnicity---is essential for prevention and targeted intervention.
Logistic regression is widely used to estimate associations between such
factors and binary outcomes like diabetes diagnosis. However, classical
maximum likelihood estimation (MLE) can produce unstable estimates in
the presence of missing data, quasi-separation, or small samples.
Bayesian logistic regression offers a robust alternative by integrating
prior information, regularizing estimates, and quantifying uncertainty
more transparently than frequentist approaches.

Bayesian hierarchical models, implemented via Markov Chain Monte Carlo
(MCMC), have been successfully applied in predicting patient health
status across diseases such as pneumonia, prostate cancer, and mental
disorders (\citeproc{ref-zeger2020}{Zeger et al. 2020}). By representing
predictive uncertainty alongside point estimates, Bayesian inference
offers a practical advantage in epidemiologic modeling where decisions
hinge on probabilistic thresholds. Beyond stability, Bayesian methods
support model checking, variable selection, and uncertainty
quantification under missingness or imputation frameworks
(\citeproc{ref-baldwin2017}{Baldwin and Larson 2017};
\citeproc{ref-kruschke2017}{Kruschke and Liddell 2017}).

Recent work has expanded Bayesian applications to disease diagnostics
and health risk modeling. For instance, Bayesian approaches have been
used to evaluate NHANES diagnostic data
(\citeproc{ref-chatzimichail2023}{Chatzimichail and Hatjimihail 2023}),
to model cardiovascular and metabolic risk (\citeproc{ref-liu2013}{Liu
et al. 2013}), and to integrate multiple data modalities such as imaging
and laboratory measures (\citeproc{ref-abdullah2022bdlhealth}{Abdullah,
Hassan, and Mustafa 2022}). Moreover, multiple imputation combined with
Bayesian modeling generates robust estimates when data are missing at
random (MAR) or not at random (MNAR) (\citeproc{ref-austin2021}{Austin
et al. 2021}).

The broader Bayesian literature emphasizes the role of priors and model
checking. Weakly informative priors, such as \(N(0, 2.5)\) for
coefficients, regularize estimation and reduce variance in small samples
(\citeproc{ref-gelman2008}{Gelman et al. 2008};
\citeproc{ref-vandeschoot2021}{Vande Schoot et al. 2021}). Tutorials
using R packages like \texttt{brms} and \texttt{blavaan} illustrate how
MCMC enables posterior inference and empirical Bayes analysis
(\citeproc{ref-klauenberg2015}{Klauenberg et al. 2015}).

Beyond standard generalized linear models, Bayesian nonparametric
regression flexibly captures nonlinearity and zero inflation common in
health data (\citeproc{ref-richardson2018bnr}{Richardson and Hartman
2018}). Bayesian Additive Regression Trees (BART) improve variable
selection in mixed-type data (\citeproc{ref-luo2024bartvs}{Luo et al.
2024}), while state-space and dynamic Bayesian models incorporate
time-varying biomarkers for longitudinal prediction
(\citeproc{ref-momeni2021covidbayes}{Momeni et al. 2021}). Bayesian
model averaging (BMA) further addresses model uncertainty by weighting
across multiple specifications (\citeproc{ref-hoeting1999bma}{Hoeting et
al. 1999}). Together, these approaches demonstrate the versatility and
growing importance of Bayesian inference in clinical and epidemiologic
modeling.

The objective of this project is to evaluate whether Bayesian inference
provides more stable and interpretable estimates of diabetes risk than
frequentist and imputation-based approaches, particularly when data
complexity or separation challenges arise. Agreement across modeling
frameworks supports the robustness of these associations and highlights
the interpretability and uncertainty quantification advantages offered
by Bayesian analysis in population health modeling
(\citeproc{ref-nchs2014}{National Center for Health Statistics (NCHS)
2014}).

\subsection{Aims}\label{aims}

The present study employs Bayesian logistic regression to predict
diabetes status and examine the relationships between diabetes and key
predictors, including body mass index (BMI), age (≥20 years), sex, and
race. Using retrospective data from the 2013--2014 NHANES survey, the
analysis accounts for the study's complex sampling design, which
involves stratification, clustering, and the oversampling of specific
subpopulations rather than simple random sampling. The Bayesian
framework is applied to address common analytical challenges such as
missing data, complete case bias, and data separation, thereby improving
the robustness and reliability of inference compared to traditional
logistic regression methods.

\section{Method}\label{method}

\subsection{Bayesian Logistic
Regression}\label{bayesian-logistic-regression}

The Bayesian framework integrates prior knowledge with observed data to
generate posterior distributions, allowing parameters to be interpreted
directly in probabilistic terms.

Unlike traditional frequentist approaches that yield single-point
estimates and p-values, Bayesian methods represent parameters as random
variables with full probability distributions.

This provides greater flexibility, incorporates parameter uncertainty,
and produces credible intervals that directly quantify the probability
that a parameter lies within a given range.

\subsection{Model Structure}\label{model-structure}

Bayesian logistic regression models the log-odds of a binary outcome as
a linear combination of predictors:

\[
\text{logit}(P(Y = 1)) = \beta_0 + \beta_1 X_1 + \beta_2 X_2 + \dots + \beta_k X_k
\]

where

\begin{itemize}
\tightlist
\item
  \(P(Y = 1)\) is the probability of the event of interest,
\item
  \(\beta_0\) is the intercept (log-odds when all predictors are zero),
  and
\item
  \(\beta_j\) represents the effect of predictor \(X_j\) on the log-odds
  of the outcome, holding other predictors constant.
\end{itemize}

In the Bayesian framework, model parameters (\(\boldsymbol{\beta}\)) are
treated as random variables and assigned prior distributions that
reflect existing knowledge or plausible ranges before observing the
data. After incorporating the observed evidence, the priors are updated
through Bayes' theorem (\citeproc{ref-deleeuw2012}{Leeuw and Klugkist
2012}; \citeproc{ref-klauenberg2015}{Klauenberg et al. 2015}):

\[
\text{Posterior} \propto \text{Likelihood} \times \text{Prior}
\]

\begin{itemize}
\tightlist
\item
  \textbf{Likelihood:} represents the probability of the observed data
  given the model parameters---it captures how well different parameter
  values explain the data.
\item
  \textbf{Prior:} expresses beliefs or existing information about the
  parameters before observing the data.
\item
  \textbf{Posterior:} combines both, representing the updated
  distribution of parameter values after accounting for the data.
\end{itemize}

This formulation allows uncertainty to propagate naturally through the
model, producing posterior distributions for each coefficient that can
be directly interpreted as probabilities.

\subsection{Prior Specification}\label{prior-specification}

Weakly informative priors were used to regularize estimation without
imposing strong assumptions:

\begin{itemize}
\tightlist
\item
  \textbf{Regression Coefficients:} \(N(0, 2.5)\), providing gentle
  regularization while allowing substantial variation in plausible
  effects (\citeproc{ref-gelman2008}{Gelman et al. 2008};
  \citeproc{ref-vandeschoot2021}{Vande Schoot et al. 2021}).
\item
  \textbf{Intercept:} Student's t-distribution prior, \(t(3, 0, 10)\)
  (\citeproc{ref-vandeschoot2013}{Schoot et al. 2013};
  \citeproc{ref-vandeschoot2021}{Vande Schoot et al. 2021}), which has

  \begin{itemize}
  \tightlist
  \item
    3 degrees of freedom (heavy tails to allow occasional large
    effects),
  \item
    mean 0 (no bias toward positive or negative effects), and
  \item
    scale 10 (broad range of possible values).
  \end{itemize}
\end{itemize}

Such priors help stabilize estimation in the presence of
multicollinearity, limited sample size, or potential outliers.

\subsection{Posterior Predictions}\label{posterior-predictions}

Posterior distributions of regression coefficients were used to estimate
the probability of the outcome for given predictor values. This allows
statements such as: \textgreater{} Given the predictors, the probability
of the outcome lies between X\% and Y\%.

Posterior predictions account for two key sources of uncertainty:

\begin{enumerate}
\def\labelenumi{\arabic{enumi}.}
\tightlist
\item
  \textbf{Parameter Uncertainty:} Variability in estimated model
  coefficients.
\item
  \textbf{Predictive Uncertainty:} Variability in possible future
  outcomes given those parameters.
\end{enumerate}

In Bayesian analysis, all unknown quantities---coefficients, means,
variances, or probabilities---are treated as random variables described
by their posterior distributions.

\subsection{Model Evaluation and
Diagnostics}\label{model-evaluation-and-diagnostics}

Model quality and convergence were assessed using standard Bayesian
diagnostics:

\begin{itemize}
\tightlist
\item
  \textbf{Posterior Sampling:} Conducted via Markov Chain Monte Carlo
  (MCMC) using the No-U-Turn Sampler (NUTS), a variant of Hamiltonian
  Monte Carlo (HMC) (\citeproc{ref-austin2021}{Austin et al. 2021}).
  Four chains were run with sufficient warm-up iterations to ensure
  convergence.
\item
  \textbf{Convergence Metrics:} The potential scale reduction factor
  (\(\hat{R}\)) and effective sample size (ESS) were used to verify
  stability and mixing across chains.
\item
  \textbf{Autocorrelation Checks:} Ensured independence between
  successive draws.
\item
  \textbf{Posterior Predictive Checks (PPCs):} Compared simulated
  outcomes to observed data to evaluate fit.
\item
  \textbf{Bayesian} \(R^2\): Quantified the proportion of variance
  explained by predictors, incorporating posterior uncertainty.
\end{itemize}

\subsection{Advantages of Bayesian Logistic
Regression}\label{advantages-of-bayesian-logistic-regression}

\begin{itemize}
\tightlist
\item
  \textbf{Uncertainty Quantification:} Produces full posterior
  distributions instead of single estimates.
\item
  \textbf{Credible Intervals:} Provide the range within which a
  parameter lies with a specified probability (e.g., 95\%).
\item
  \textbf{Flexible Priors:} Allow integration of expert knowledge or
  findings from prior studies.
\item
  \textbf{Probabilistic Predictions:} Posterior predictive distributions
  yield direct probabilities for new or future observations.
\item
  \textbf{Model Evaluation:} PPCs assess how well simulated outcomes
  reproduce observed data.
\end{itemize}

\section{Analysis and Results}\label{analysis-and-results}

\subsection{Data Preparation}\label{data-preparation}

This study used publicly available 2013--2014 NHANES data published by
the CDC's National Center for Health Statistics
(\citeproc{ref-nchs2014}{National Center for Health Statistics (NCHS)
2014}). Three component files were utilized: \texttt{DEMO\_H}
(demographics), \texttt{BMX\_H} (body measures), and \texttt{DIQ\_H}
(diabetes questionnaire). Each file was imported in \texttt{.XPT} format
using the \textbf{\texttt{haven}} package in \textbf{R}, and merged
using the unique participant identifier \texttt{SEQN} to create a single
adult analytic dataset (age ≥ 20 years).

All variables were coerced to consistent numeric or factor types prior
to merging to ensure atomic columns suitable for survey-weighted
analysis and modeling. The use of \texttt{SEQN} preserved respondent
integrity across datasets and ensured accurate record linkage. This
preprocessing step standardized variable formats and minimized
inconsistencies between files.

Data wrangling, cleaning, and merging were performed in \textbf{R} using
a combination of base functions and tidyverse packages. Bayesian
logistic regression modeling was later implemented using the
\textbf{\texttt{brms}} interface to \textbf{Stan}, allowing
probabilistic inference within a reproducible workflow that accommodated
the NHANES complex survey design and missing data considerations.

\subsubsection{Data Import and Merging}\label{data-import-and-merging}

\begin{Shaded}
\begin{Highlighting}[]
\NormalTok{merged\_data }\OtherTok{\textless{}{-}} \FunctionTok{readRDS}\NormalTok{(}\StringTok{"data/merged\_2013\_2014.rds"}\NormalTok{)}

\NormalTok{merged\_n }\OtherTok{\textless{}{-}} \FunctionTok{nrow}\NormalTok{(merged\_data)}
\end{Highlighting}
\end{Shaded}

The merged dataset contains 10,175 participants. It integrates
demographic, examination, and diabetes questionnaire data. We then
restrict the sample to adults (age ≥ 20) to define the analytic cohort
used in subsequent analyses. A small proportion of records have missing
values in BMI and diabetes status, which will be addressed later through
multiple imputation.

\begin{longtable}[]{@{}rrrrr@{}}
\caption{Preview of merged NHANES 2013--2014 dataset limited to analysis
variables (source columns only).}\tabularnewline
\toprule\noalign{}
RIDAGEYR & BMXBMI & RIAGENDR & RIDRETH1 & DIQ010 \\
\midrule\noalign{}
\endfirsthead
\toprule\noalign{}
RIDAGEYR & BMXBMI & RIAGENDR & RIDRETH1 & DIQ010 \\
\midrule\noalign{}
\endhead
\bottomrule\noalign{}
\endlastfoot
69 & 26.7 & 1 & 4 & 1 \\
54 & 28.6 & 1 & 3 & 1 \\
72 & 28.9 & 1 & 3 & 1 \\
9 & 17.1 & 1 & 3 & 2 \\
73 & 19.7 & 2 & 3 & 2 \\
56 & 41.7 & 1 & 1 & 2 \\
0 & NA & 1 & 3 & NA \\
61 & 35.7 & 2 & 3 & 2 \\
42 & NA & 1 & 2 & 2 \\
56 & 26.5 & 2 & 3 & 2 \\
\end{longtable}

\subsubsection{Variable Definitions}\label{variable-definitions}

\begin{itemize}
\item
  \textbf{Response Variable:}\\
  \texttt{diabetes\_dx} (binary) represents a Type 2 diabetes diagnosis,
  excluding gestational diabetes. It was derived from \texttt{DIQ010}
  (``Doctor told you have diabetes''), while \texttt{DIQ050} (insulin
  use) was excluded to prevent treatment-related confounding.
\item
  \textbf{Predictor Variables:}

  \begin{itemize}
  \tightlist
  \item
    \texttt{BMXBMI} -- Body Mass Index (kg/m\^{}2), treated as
    continuous and categorized into six BMI classes
    (\texttt{bmi\_cat}).\\
  \item
    \texttt{RIDAGEYR} -- Age (continuous, 20--80 years)\\
  \item
    \texttt{RIAGENDR} -- Sex (factor, two levels)\\
  \item
    \texttt{RIDRETH1} -- Ethnicity (factor, five levels)
  \end{itemize}
\end{itemize}

\begin{Shaded}
\begin{Highlighting}[]
\CommentTok{\# {-}{-}{-}{-}{-}{-}{-}{-}{-}{-}{-}{-}{-}{-}{-}{-}{-}{-}{-}{-}{-}{-}{-}{-}{-}{-}{-}{-}{-}}
\CommentTok{\# Variable descriptions (with \textasciigrave{}code\textasciigrave{} formatting for names)}
\CommentTok{\# {-}{-}{-}{-}{-}{-}{-}{-}{-}{-}{-}{-}{-}{-}{-}{-}{-}{-}{-}{-}{-}{-}{-}{-}{-}{-}{-}{-}{-}}
\NormalTok{var\_tbl }\OtherTok{\textless{}{-}} \FunctionTok{tribble}\NormalTok{(}
  \SpecialCharTok{\textasciitilde{}}\NormalTok{Variable,      }\SpecialCharTok{\textasciitilde{}}\NormalTok{Description,                                                                                                   }\SpecialCharTok{\textasciitilde{}}\NormalTok{Type,         }\SpecialCharTok{\textasciitilde{}}\NormalTok{Origin,}
  \StringTok{"\textasciigrave{}diabetes\_dx\textasciigrave{}"}\NormalTok{,}\StringTok{"Type 2 diabetes diagnosis (1 = Yes, 0 = No) derived from \textasciigrave{}DIQ010\textasciigrave{}; gestational diabetes excluded."}\NormalTok{,           }\StringTok{"Categorical"}\NormalTok{, }\StringTok{"Derived from \textasciigrave{}DIQ010\textasciigrave{}"}\NormalTok{,}
  \StringTok{"\textasciigrave{}age\textasciigrave{}"}\NormalTok{,        }\StringTok{"Age in years."}\NormalTok{,                                                                                                }\StringTok{"Continuous"}\NormalTok{,  }\StringTok{"NHANES \textasciigrave{}RIDAGEYR\textasciigrave{}"}\NormalTok{,}
  \StringTok{"\textasciigrave{}bmi\textasciigrave{}"}\NormalTok{,        }\StringTok{"Body Mass Index (kg/m\^{}2) computed from measured height and weight."}\NormalTok{,                                           }\StringTok{"Continuous"}\NormalTok{,  }\StringTok{"NHANES \textasciigrave{}BMXBMI\textasciigrave{}"}\NormalTok{,}
  \StringTok{"\textasciigrave{}bmi\_cat\textasciigrave{}"}\NormalTok{,    }\StringTok{"BMI categories: Underweight, Normal, Overweight, Obesity I–III (\textasciigrave{}Normal\textasciigrave{} is reference in models)."}\NormalTok{,            }\StringTok{"Categorical"}\NormalTok{, }\StringTok{"Derived from \textasciigrave{}bmi\textasciigrave{}"}\NormalTok{,}
  \StringTok{"\textasciigrave{}sex\textasciigrave{}"}\NormalTok{,        }\StringTok{"Sex of participant (\textasciigrave{}Male\textasciigrave{}, \textasciigrave{}Female\textasciigrave{})."}\NormalTok{,                                                                       }\StringTok{"Categorical"}\NormalTok{, }\StringTok{"NHANES \textasciigrave{}RIAGENDR\textasciigrave{}"}\NormalTok{,}
  \StringTok{"\textasciigrave{}race\textasciigrave{}"}\NormalTok{, }\StringTok{"race/Ethnicity collapsed to four levels: White, Black, Hispanic, Other."}\NormalTok{, }\StringTok{"Categorical"}\NormalTok{, }\StringTok{"Derived from \textasciigrave{}RIDRETH1\textasciigrave{}"}\NormalTok{,}
  \StringTok{"\textasciigrave{}WTMEC2YR\textasciigrave{}"}\NormalTok{,   }\StringTok{"Examination sample weight for Mobile Examination Center participants."}\NormalTok{,                                        }\StringTok{"Weight"}\NormalTok{,      }\StringTok{"NHANES design"}\NormalTok{,}
  \StringTok{"\textasciigrave{}SDMVPSU\textasciigrave{}"}\NormalTok{,    }\StringTok{"Primary Sampling Unit used for variance estimation in the complex survey design."}\NormalTok{,                             }\StringTok{"Design"}\NormalTok{,      }\StringTok{"NHANES design"}\NormalTok{,}
  \StringTok{"\textasciigrave{}SDMVSTRA\textasciigrave{}"}\NormalTok{,   }\StringTok{"Stratum identifier used to define strata for the complex survey design."}\NormalTok{,                                      }\StringTok{"Design"}\NormalTok{,      }\StringTok{"NHANES design"}\NormalTok{,}
  \StringTok{"\textasciigrave{}age\_c\textasciigrave{}"}\NormalTok{,      }\StringTok{"Centered and standardized age (z{-}score)."}\NormalTok{,                                                                     }\StringTok{"Continuous"}\NormalTok{,  }\StringTok{"Derived from \textasciigrave{}age\textasciigrave{}"}\NormalTok{,}
  \StringTok{"\textasciigrave{}bmi\_c\textasciigrave{}"}\NormalTok{,      }\StringTok{"Centered and standardized BMI (z{-}score)."}\NormalTok{,                                                                     }\StringTok{"Continuous"}\NormalTok{,  }\StringTok{"Derived from \textasciigrave{}bmi\textasciigrave{}"}
\NormalTok{)}

\FunctionTok{kbl}\NormalTok{(}
\NormalTok{  var\_tbl,}
  \AttributeTok{caption =} \StringTok{"Variable Descriptions: Adult Analytic Dataset"}\NormalTok{,}
  \AttributeTok{align =} \FunctionTok{c}\NormalTok{(}\StringTok{"l"}\NormalTok{,}\StringTok{"l"}\NormalTok{,}\StringTok{"l"}\NormalTok{,}\StringTok{"l"}\NormalTok{),}
  \AttributeTok{escape =} \ConstantTok{FALSE}
\NormalTok{) }\SpecialCharTok{\%\textgreater{}\%}
  \FunctionTok{kable\_styling}\NormalTok{(}\AttributeTok{full\_width =} \ConstantTok{FALSE}\NormalTok{, }\AttributeTok{position =} \StringTok{"center"}\NormalTok{, }\AttributeTok{bootstrap\_options =} \FunctionTok{c}\NormalTok{(}\StringTok{"striped"}\NormalTok{,}\StringTok{"hover"}\NormalTok{)) }\SpecialCharTok{\%\textgreater{}\%}
  \FunctionTok{group\_rows}\NormalTok{(}\StringTok{"Analysis variables"}\NormalTok{, }\DecValTok{1}\NormalTok{, }\DecValTok{6}\NormalTok{) }\SpecialCharTok{\%\textgreater{}\%}              \CommentTok{\# \textless{}{-}{-} updated range (now 6 analysis rows)}
  \FunctionTok{group\_rows}\NormalTok{(}\StringTok{"Survey design variables"}\NormalTok{, }\DecValTok{7}\NormalTok{, }\DecValTok{9}\NormalTok{) }\SpecialCharTok{\%\textgreater{}\%}
  \FunctionTok{group\_rows}\NormalTok{(}\StringTok{"Derived variables"}\NormalTok{, }\DecValTok{10}\NormalTok{, }\DecValTok{11}\NormalTok{)}
\end{Highlighting}
\end{Shaded}

\begin{table}
\centering
\caption{\label{tab:variables-table}Variable Descriptions: Adult Analytic Dataset}
\centering
\begin{tabular}[t]{l|l|l|l}
\hline
Variable & Description & Type & Origin\\
\hline
\multicolumn{4}{l}{\textbf{Analysis variables}}\\
\hline
\hspace{1em}`diabetes_dx` & Type 2 diabetes diagnosis (1 = Yes, 0 = No) derived from `DIQ010`; gestational diabetes excluded. & Categorical & Derived from `DIQ010`\\
\hline
\hspace{1em}`age` & Age in years. & Continuous & NHANES `RIDAGEYR`\\
\hline
\hspace{1em}`bmi` & Body Mass Index (kg/m^2) computed from measured height and weight. & Continuous & NHANES `BMXBMI`\\
\hline
\hspace{1em}`bmi_cat` & BMI categories: Underweight, Normal, Overweight, Obesity I–III (`Normal` is reference in models). & Categorical & Derived from `bmi`\\
\hline
\hspace{1em}`sex` & Sex of participant (`Male`, `Female`). & Categorical & NHANES `RIAGENDR`\\
\hline
\hspace{1em}`race` & race/Ethnicity collapsed to four levels: White, Black, Hispanic, Other. & Categorical & Derived from `RIDRETH1`\\
\hline
\multicolumn{4}{l}{\textbf{Survey design variables}}\\
\hline
\hspace{1em}`WTMEC2YR` & Examination sample weight for Mobile Examination Center participants. & Weight & NHANES design\\
\hline
\hspace{1em}`SDMVPSU` & Primary Sampling Unit used for variance estimation in the complex survey design. & Design & NHANES design\\
\hline
\hspace{1em}`SDMVSTRA` & Stratum identifier used to define strata for the complex survey design. & Design & NHANES design\\
\hline
\multicolumn{4}{l}{\textbf{Derived variables}}\\
\hline
\hspace{1em}`age_c` & Centered and standardized age (z-score). & Continuous & Derived from `age`\\
\hline
\hspace{1em}`bmi_c` & Centered and standardized BMI (z-score). & Continuous & Derived from `bmi`\\
\hline
\end{tabular}
\end{table}

\subsubsection{Study Design and Survey-Weighted
Analysis}\label{study-design-and-survey-weighted-analysis}

The National Health and Nutrition Examination Survey (NHANES) employs a
complex, multistage probability sampling design with stratification,
clustering, and oversampling of specific demographic groups (for
example, racial/ethnic minorities and older adults) to produce
nationally representative estimates of the U.S. population.

Survey design variables --- primary sampling units (\texttt{SDMVPSU}),
strata (\texttt{SDMVSTRA}), and examination sample weights
(\texttt{WTMEC2YR}) --- were retained to account for this complex
design. These variables were applied to adjust for unequal probabilities
of selection, nonresponse, and oversampling, ensuring valid standard
errors, unbiased prevalence estimates, and generalizable
population-level inference.

A survey-weighted logistic regression model was used to evaluate the
association between diabetes status (\texttt{diabetes\_dx}, binary
outcome) and key predictors: body mass index (\texttt{bmi}), age
(\texttt{age}), sex (\texttt{sex}), and race/ethnicity (\texttt{race}).
Diabetes was defined using \texttt{DIQ010} (``Doctor told you have
diabetes'') and coded as 0/1, with \texttt{DIQ050} (insulin use)
excluded to avoid treatment-related confounding.

Covariates included:\\
- \texttt{age} (continuous; centered as \texttt{age\_c}, categorized
20--80 years)\\
- \texttt{bmi} (continuous; centered as \texttt{bmi\_c}, and categorized
by BMI class \texttt{bmi\_cat})\\
- \texttt{sex} (male, female)\\
- \texttt{race} (four ethnicity levels: White, Black, Hispanic, Other)

This approach accounts for NHANES' complex sampling design, producing
unbiased parameter estimates and valid inference for U.S. adults.

\begin{longtable}[]{@{}
  >{\raggedright\arraybackslash}p{(\linewidth - 2\tabcolsep) * \real{0.3333}}
  >{\raggedright\arraybackslash}p{(\linewidth - 2\tabcolsep) * \real{0.6667}}@{}}
\toprule\noalign{}
\begin{minipage}[b]{\linewidth}\raggedright
Step
\end{minipage} & \begin{minipage}[b]{\linewidth}\raggedright
Description
\end{minipage} \\
\midrule\noalign{}
\endhead
\bottomrule\noalign{}
\endlastfoot
\textbf{Weighting} & Used the \textbf{\texttt{survey}} package to
calculate weighted means for key variables (e.g., age and diabetes
status) and to estimate design effects and effective sample size for the
complex survey design. \\
\textbf{Standardization} & Centered and standardized BMI and age
(\texttt{bmi\_c}, \texttt{age\_c}) for use in regression models. \\
\textbf{Age Categorization} & Not implemented in the analytic dataset
(continuous \texttt{age} retained). Reference retained for potential
descriptive grouping (20--\textless30, 30--\textless40, 40--\textless50,
50--\textless60, 60--\textless70, 70--80). \\
\textbf{BMI Categorization} & Recoded as: \textless18.5 (Underweight),
18.5--\textless25 (Normal), 25--\textless30 (Overweight),
30--\textless35 (Obesity I), 35--\textless40 (Obesity II), ≥40 (Obesity
III). \\
\textbf{Ethnicity Recoding} & \texttt{RIDRETH1} recoded as: 1 = Mexican
American, 2 = Other Hispanic, 3 = Non-Hispanic White, 4 = Non-Hispanic
Black, 5 = Other/Multi; then \texttt{NH\ White} set as the reference
level (five analytical levels retained). \\
\textbf{Special Codes} & Transformed nonresponse codes (e.g., 3, 7, 9)
to \texttt{NA}. These missing codes were evaluated for potential
nonrandom patterns (MAR/MNAR). \\
\textbf{Missing Data} & Retained and visualized missing values
(primarily in BMI and diabetes status) to assess their pattern and
informativeness before multiple imputation. \\
\textbf{Final Dataset} & Created the cleaned analytic dataset
(\texttt{adult}) using \emph{Non-Hispanic White} and \emph{Male} as
reference groups for modeling, preserving NHANES survey design variables
(\texttt{WTMEC2YR}, \texttt{SDMVPSU}, \texttt{SDMVSTRA}). \\
\end{longtable}

\subsubsection{Adult Cohort Definition}\label{adult-cohort-definition}

\begin{Shaded}
\begin{Highlighting}[]
\CommentTok{\# NHANES survey design object for the adult analytic cohort}

\NormalTok{nhanes\_design\_adult }\OtherTok{\textless{}{-}}\NormalTok{ survey}\SpecialCharTok{::}\FunctionTok{svydesign}\NormalTok{(}
\AttributeTok{id      =} \SpecialCharTok{\textasciitilde{}}\NormalTok{SDMVPSU,}
\AttributeTok{strata  =} \SpecialCharTok{\textasciitilde{}}\NormalTok{SDMVSTRA,}
\AttributeTok{weights =} \SpecialCharTok{\textasciitilde{}}\NormalTok{WTMEC2YR,}
\AttributeTok{nest    =} \ConstantTok{TRUE}\NormalTok{,}
\AttributeTok{data    =}\NormalTok{ adult}
\NormalTok{)}

\CommentTok{\# Quick weighted checks}

\NormalTok{survey}\SpecialCharTok{::}\FunctionTok{svymean}\NormalTok{(}\SpecialCharTok{\textasciitilde{}}\NormalTok{age, nhanes\_design\_adult, }\AttributeTok{na.rm =} \ConstantTok{TRUE}\NormalTok{)}
\end{Highlighting}
\end{Shaded}

\begin{verbatim}
      mean     SE
age 47.496 0.3805
\end{verbatim}

\begin{Shaded}
\begin{Highlighting}[]
\NormalTok{survey}\SpecialCharTok{::}\FunctionTok{svymean}\NormalTok{(}\SpecialCharTok{\textasciitilde{}}\NormalTok{diabetes\_dx, nhanes\_design\_adult, }\AttributeTok{na.rm =} \ConstantTok{TRUE}\NormalTok{)}
\end{Highlighting}
\end{Shaded}

\begin{verbatim}
                mean     SE
diabetes_dx 0.089016 0.0048
\end{verbatim}

\begin{Shaded}
\begin{Highlighting}[]
\CommentTok{\# Design effect and effective sample size for \textasciigrave{}diabetes\_dx\textasciigrave{}}

\NormalTok{v\_hat }\OtherTok{\textless{}{-}} \FunctionTok{as.numeric}\NormalTok{(survey}\SpecialCharTok{::}\FunctionTok{svyvar}\NormalTok{(}\SpecialCharTok{\textasciitilde{}}\NormalTok{diabetes\_dx, nhanes\_design\_adult, }\AttributeTok{na.rm =} \ConstantTok{TRUE}\NormalTok{))}
\NormalTok{p\_hat }\OtherTok{\textless{}{-}} \FunctionTok{mean}\NormalTok{(adult}\SpecialCharTok{$}\NormalTok{diabetes\_dx, }\AttributeTok{na.rm =} \ConstantTok{TRUE}\NormalTok{)}
\NormalTok{n\_obs }\OtherTok{\textless{}{-}} \FunctionTok{nrow}\NormalTok{(adult)}
\NormalTok{v\_srs }\OtherTok{\textless{}{-}}\NormalTok{ p\_hat }\SpecialCharTok{*}\NormalTok{ (}\DecValTok{1} \SpecialCharTok{{-}}\NormalTok{ p\_hat) }\SpecialCharTok{/}\NormalTok{ n\_obs}
\NormalTok{deff  }\OtherTok{\textless{}{-}}\NormalTok{ v\_hat }\SpecialCharTok{/}\NormalTok{ v\_srs}

\NormalTok{n\_total }\OtherTok{\textless{}{-}} \FunctionTok{sum}\NormalTok{(}\FunctionTok{weights}\NormalTok{(nhanes\_design\_adult), }\AttributeTok{na.rm =} \ConstantTok{TRUE}\NormalTok{)}
\NormalTok{ess     }\OtherTok{\textless{}{-}} \FunctionTok{as.numeric}\NormalTok{(n\_total }\SpecialCharTok{/}\NormalTok{ deff)}

\FunctionTok{cat}\NormalTok{(}\StringTok{"Design effect for diabetes\_dx:"}\NormalTok{, }\FunctionTok{round}\NormalTok{(deff, }\DecValTok{2}\NormalTok{), }\StringTok{"}\SpecialCharTok{\textbackslash{}n}\StringTok{"}\NormalTok{)}
\end{Highlighting}
\end{Shaded}

\begin{verbatim}
Design effect for diabetes_dx: 4759.91 
\end{verbatim}

\begin{Shaded}
\begin{Highlighting}[]
\FunctionTok{cat}\NormalTok{(}\StringTok{"Effective sample size for diabetes\_dx:"}\NormalTok{, }\FunctionTok{round}\NormalTok{(ess), }\StringTok{"}\SpecialCharTok{\textbackslash{}n}\StringTok{"}\NormalTok{)}
\end{Highlighting}
\end{Shaded}

\begin{verbatim}
Effective sample size for diabetes_dx: 48142 
\end{verbatim}

Descriptive statistics for continuous and categorical variables are
presented below.

\begin{Shaded}
\begin{Highlighting}[]
\CommentTok{\# Keep only analytic variables for Table 1}
\NormalTok{tbl1\_dat }\OtherTok{\textless{}{-}}\NormalTok{ adult }\SpecialCharTok{\%\textgreater{}\%}
  \FunctionTok{transmute}\NormalTok{(}
\NormalTok{    age,}
\NormalTok{    bmi,}
\NormalTok{    bmi\_cat,}
\NormalTok{    sex,}
\NormalTok{    race,}
    \AttributeTok{diabetes\_dx =} \FunctionTok{factor}\NormalTok{(diabetes\_dx, }\AttributeTok{levels =} \FunctionTok{c}\NormalTok{(}\DecValTok{0}\NormalTok{, }\DecValTok{1}\NormalTok{), }\AttributeTok{labels =} \FunctionTok{c}\NormalTok{(}\StringTok{"No"}\NormalTok{, }\StringTok{"Yes"}\NormalTok{))}
\NormalTok{  )}

\CommentTok{\# Continuous summaries: N, missing, mean, sd, min, max}
\NormalTok{cont\_vars }\OtherTok{\textless{}{-}} \FunctionTok{c}\NormalTok{(}\StringTok{"age"}\NormalTok{, }\StringTok{"bmi"}\NormalTok{)}

\NormalTok{cont\_sum }\OtherTok{\textless{}{-}}\NormalTok{ tbl1\_dat }\SpecialCharTok{\%\textgreater{}\%}
  \FunctionTok{select}\NormalTok{(}\FunctionTok{all\_of}\NormalTok{(cont\_vars)) }\SpecialCharTok{\%\textgreater{}\%}
  \FunctionTok{pivot\_longer}\NormalTok{(}\FunctionTok{everything}\NormalTok{(), }\AttributeTok{names\_to =} \StringTok{"Variable"}\NormalTok{, }\AttributeTok{values\_to =} \StringTok{"value"}\NormalTok{) }\SpecialCharTok{\%\textgreater{}\%}
  \FunctionTok{group\_by}\NormalTok{(Variable) }\SpecialCharTok{\%\textgreater{}\%}
  \FunctionTok{summarise}\NormalTok{(}
    \AttributeTok{N       =} \FunctionTok{sum}\NormalTok{(}\SpecialCharTok{!}\FunctionTok{is.na}\NormalTok{(value)),}
    \AttributeTok{Missing =} \FunctionTok{sum}\NormalTok{(}\FunctionTok{is.na}\NormalTok{(value)),}
    \AttributeTok{Mean    =} \FunctionTok{round}\NormalTok{(}\FunctionTok{mean}\NormalTok{(value, }\AttributeTok{na.rm =} \ConstantTok{TRUE}\NormalTok{), }\DecValTok{2}\NormalTok{),}
    \AttributeTok{SD      =} \FunctionTok{round}\NormalTok{(}\FunctionTok{sd}\NormalTok{(value, }\AttributeTok{na.rm =} \ConstantTok{TRUE}\NormalTok{), }\DecValTok{2}\NormalTok{),}
    \AttributeTok{Min     =} \FunctionTok{round}\NormalTok{(}\FunctionTok{min}\NormalTok{(value, }\AttributeTok{na.rm =} \ConstantTok{TRUE}\NormalTok{), }\DecValTok{1}\NormalTok{),}
    \AttributeTok{Max     =} \FunctionTok{round}\NormalTok{(}\FunctionTok{max}\NormalTok{(value, }\AttributeTok{na.rm =} \ConstantTok{TRUE}\NormalTok{), }\DecValTok{1}\NormalTok{),}
    \AttributeTok{.groups =} \StringTok{"drop"}
\NormalTok{  )}

\CommentTok{\# Categorical summaries: counts and percents}
\NormalTok{cat\_vars }\OtherTok{\textless{}{-}} \FunctionTok{c}\NormalTok{(}\StringTok{"sex"}\NormalTok{, }\StringTok{"race"}\NormalTok{, }\StringTok{"diabetes\_dx"}\NormalTok{, }\StringTok{"bmi\_cat"}\NormalTok{)}

\NormalTok{cat\_sum }\OtherTok{\textless{}{-}}\NormalTok{ tbl1\_dat }\SpecialCharTok{\%\textgreater{}\%}
  \FunctionTok{mutate}\NormalTok{(}\FunctionTok{across}\NormalTok{(}\FunctionTok{all\_of}\NormalTok{(cat\_vars),}
                \SpecialCharTok{\textasciitilde{}}\NormalTok{ forcats}\SpecialCharTok{::}\FunctionTok{fct\_explicit\_na}\NormalTok{(}\FunctionTok{as.factor}\NormalTok{(.x), }\AttributeTok{na\_level =} \StringTok{"(Missing)"}\NormalTok{))) }\SpecialCharTok{\%\textgreater{}\%}
  \FunctionTok{select}\NormalTok{(}\FunctionTok{all\_of}\NormalTok{(cat\_vars)) }\SpecialCharTok{\%\textgreater{}\%}
  \FunctionTok{pivot\_longer}\NormalTok{(}\FunctionTok{everything}\NormalTok{(), }\AttributeTok{names\_to =} \StringTok{"Variable"}\NormalTok{, }\AttributeTok{values\_to =} \StringTok{"Level"}\NormalTok{) }\SpecialCharTok{\%\textgreater{}\%}
  \FunctionTok{count}\NormalTok{(Variable, Level, }\AttributeTok{name =} \StringTok{"n"}\NormalTok{) }\SpecialCharTok{\%\textgreater{}\%}
  \FunctionTok{group\_by}\NormalTok{(Variable) }\SpecialCharTok{\%\textgreater{}\%}
  \FunctionTok{mutate}\NormalTok{(}\AttributeTok{pct =} \FunctionTok{round}\NormalTok{(}\DecValTok{100} \SpecialCharTok{*}\NormalTok{ n }\SpecialCharTok{/} \FunctionTok{sum}\NormalTok{(n), }\DecValTok{1}\NormalTok{)) }\SpecialCharTok{\%\textgreater{}\%}
  \FunctionTok{ungroup}\NormalTok{() }\SpecialCharTok{\%\textgreater{}\%}
  \FunctionTok{arrange}\NormalTok{(Variable, }\FunctionTok{desc}\NormalTok{(n))}

\CommentTok{\# Render tables}
\FunctionTok{kable}\NormalTok{(cont\_sum,}
      \AttributeTok{caption =} \StringTok{"Table 1a. Continuous variables (age, BMI): N, missing, mean (SD), range."}\NormalTok{) }\SpecialCharTok{\%\textgreater{}\%}
  \FunctionTok{kable\_styling}\NormalTok{(}\AttributeTok{full\_width =} \ConstantTok{FALSE}\NormalTok{)}
\end{Highlighting}
\end{Shaded}

\begin{longtable}[t]{lrrrrrr}
\caption{\label{tab:tbl1-analytic}Table 1a. Continuous variables (age, BMI): N, missing, mean (SD), range.}\\
\toprule
Variable & N & Missing & Mean & SD & Min & Max\\
\midrule
age & 5769 & 0 & 49.11 & 17.56 & 20.0 & 80.0\\
bmi & 5520 & 249 & 29.10 & 7.15 & 14.1 & 82.9\\
\bottomrule
\end{longtable}

\begin{Shaded}
\begin{Highlighting}[]
\FunctionTok{kable}\NormalTok{(cat\_sum,}
      \AttributeTok{caption =} \StringTok{"Table 1b. Categorical variables (sex, race, diabetes\_dx, bmi\_cat): counts and percentages."}\NormalTok{) }\SpecialCharTok{\%\textgreater{}\%}
  \FunctionTok{kable\_styling}\NormalTok{(}\AttributeTok{full\_width =} \ConstantTok{FALSE}\NormalTok{)}
\end{Highlighting}
\end{Shaded}

\begin{longtable}[t]{llrr}
\caption{\label{tab:tbl1-analytic}Table 1b. Categorical variables (sex, race, diabetes_dx, bmi_cat): counts and percentages.}\\
\toprule
Variable & Level & n & pct\\
\midrule
bmi\_cat & 25–<30 & 1768 & 30.6\\
bmi\_cat & 18.5–<25 & 1579 & 27.4\\
bmi\_cat & 30–<35 & 1145 & 19.8\\
bmi\_cat & 35–<40 & 519 & 9.0\\
bmi\_cat & ≥40 & 419 & 7.3\\
\addlinespace
bmi\_cat & (Missing) & 249 & 4.3\\
bmi\_cat & <18.5 & 90 & 1.6\\
diabetes\_dx & No & 4974 & 86.2\\
diabetes\_dx & Yes & 618 & 10.7\\
diabetes\_dx & (Missing) & 177 & 3.1\\
\addlinespace
race & NH White & 2472 & 42.8\\
race & NH Black & 1177 & 20.4\\
race & Other/Multi & 845 & 14.6\\
race & Mexican American & 767 & 13.3\\
race & Other Hispanic & 508 & 8.8\\
\addlinespace
sex & Female & 3011 & 52.2\\
sex & Male & 2758 & 47.8\\
\bottomrule
\end{longtable}

Table 1a and 1b summarize the analytic variables included in subsequent
models. Mean age and BMI values indicate an adult cohort spanning a wide
range of body composition, while categorical summaries confirm balanced
sex representation and sufficient sample sizes across race/ethnicity
categories. These variables were standardized and used as predictors in
all modeling frameworks (analytic cohort N = 5,769 adults ≥ 20 years).

\begin{Shaded}
\begin{Highlighting}[]
\NormalTok{adult\_n }\OtherTok{\textless{}{-}} \FunctionTok{nrow}\NormalTok{(adult)}
\end{Highlighting}
\end{Shaded}

\begin{longtable}[]{@{}
  >{\raggedleft\arraybackslash}p{(\linewidth - 22\tabcolsep) * \real{0.0741}}
  >{\raggedleft\arraybackslash}p{(\linewidth - 22\tabcolsep) * \real{0.0833}}
  >{\raggedleft\arraybackslash}p{(\linewidth - 22\tabcolsep) * \real{0.0833}}
  >{\raggedleft\arraybackslash}p{(\linewidth - 22\tabcolsep) * \real{0.1111}}
  >{\raggedleft\arraybackslash}p{(\linewidth - 22\tabcolsep) * \real{0.0463}}
  >{\raggedleft\arraybackslash}p{(\linewidth - 22\tabcolsep) * \real{0.0370}}
  >{\raggedright\arraybackslash}p{(\linewidth - 22\tabcolsep) * \real{0.0648}}
  >{\raggedright\arraybackslash}p{(\linewidth - 22\tabcolsep) * \real{0.1574}}
  >{\raggedleft\arraybackslash}p{(\linewidth - 22\tabcolsep) * \real{0.0648}}
  >{\raggedleft\arraybackslash}p{(\linewidth - 22\tabcolsep) * \real{0.0926}}
  >{\raggedleft\arraybackslash}p{(\linewidth - 22\tabcolsep) * \real{0.1019}}
  >{\raggedright\arraybackslash}p{(\linewidth - 22\tabcolsep) * \real{0.0833}}@{}}

\caption{\label{tbl-adult}Excerpt of the cleaned NHANES 2013--2014 adult
cohort (age ≥ 20; N = 5,769) with derived and standardized variables.}

\tabularnewline

\toprule\noalign{}
\begin{minipage}[b]{\linewidth}\raggedleft
SDMVPSU
\end{minipage} & \begin{minipage}[b]{\linewidth}\raggedleft
SDMVSTRA
\end{minipage} & \begin{minipage}[b]{\linewidth}\raggedleft
WTMEC2YR
\end{minipage} & \begin{minipage}[b]{\linewidth}\raggedleft
diabetes\_dx
\end{minipage} & \begin{minipage}[b]{\linewidth}\raggedleft
bmi
\end{minipage} & \begin{minipage}[b]{\linewidth}\raggedleft
age
\end{minipage} & \begin{minipage}[b]{\linewidth}\raggedright
sex
\end{minipage} & \begin{minipage}[b]{\linewidth}\raggedright
race
\end{minipage} & \begin{minipage}[b]{\linewidth}\raggedleft
DIQ050
\end{minipage} & \begin{minipage}[b]{\linewidth}\raggedleft
age\_c
\end{minipage} & \begin{minipage}[b]{\linewidth}\raggedleft
bmi\_c
\end{minipage} & \begin{minipage}[b]{\linewidth}\raggedright
bmi\_cat
\end{minipage} \\
\midrule\noalign{}
\endhead
\bottomrule\noalign{}
\endlastfoot
1 & 112 & 13481.04 & 1 & 26.7 & 69 & Male & NH Black & 1 & 1.1324183 &
-0.3358861 & 25--\textless30 \\
1 & 108 & 24471.77 & 1 & 28.6 & 54 & Male & NH White & 1 & 0.2783598 &
-0.0702810 & 25--\textless30 \\
1 & 109 & 57193.29 & 1 & 28.9 & 72 & Male & NH White & 1 & 1.3032300 &
-0.0283434 & 25--\textless30 \\
2 & 116 & 65541.87 & 0 & 19.7 & 73 & Female & NH White & 2 & 1.3601672 &
-1.3144311 & 18.5--\textless25 \\
1 & 111 & 25344.99 & 0 & 41.7 & 56 & Male & Mexican American & 2 &
0.3922343 & 1.7609961 & ≥40 \\
1 & 114 & 61758.65 & 0 & 35.7 & 61 & Female & NH White & 2 & 0.6769204 &
0.9222432 & 35--\textless40 \\

\end{longtable}

As shown in Table~\ref{tbl-adult}, the analytic adult cohort (N = 5,769)
includes standardized variables for age and BMI (\texttt{age\_c},
\texttt{bmi\_c}), categorical indicators for sex and race/ethnicity
(\texttt{race}), and a binary doctor-diagnosed diabetes variable
(\texttt{diabetes\_dx}).

\begin{Shaded}
\begin{Highlighting}[]
\CommentTok{\# Textual structure and preview}
\FunctionTok{str}\NormalTok{(adult)}
\end{Highlighting}
\end{Shaded}

\begin{verbatim}
'data.frame':   5769 obs. of  12 variables:
 $ SDMVPSU    : num  1 1 1 2 1 1 2 1 2 2 ...
 $ SDMVSTRA   : num  112 108 109 116 111 114 106 112 112 113 ...
 $ WTMEC2YR   : num  13481 24472 57193 65542 25345 ...
 $ diabetes_dx: num  1 1 1 0 0 0 0 0 0 0 ...
 $ bmi        : num  26.7 28.6 28.9 19.7 41.7 35.7 NA 26.5 22 20.3 ...
 $ age        : num  69 54 72 73 56 61 42 56 65 26 ...
 $ sex        : Factor w/ 2 levels "Male","Female": 1 1 1 2 1 2 1 2 1 2 ...
 $ race       : Factor w/ 5 levels "NH White","Mexican American",..: 4 1 1 1 2 1 3 1 1 1 ...
 $ DIQ050     : num  1 1 1 2 2 2 2 2 2 2 ...
 $ age_c      : num  1.132 0.278 1.303 1.36 0.392 ...
 $ bmi_c      : num  -0.3359 -0.0703 -0.0283 -1.3144 1.761 ...
 $ bmi_cat    : Factor w/ 6 levels "<18.5","18.5–<25",..: 3 3 3 2 6 5 NA 3 2 2 ...
\end{verbatim}

\begin{Shaded}
\begin{Highlighting}[]
\CommentTok{\# Visual structure and type overview}
\FunctionTok{plot\_intro}\NormalTok{(adult, }\AttributeTok{title =} \StringTok{"Adult Dataset: Variable Types and Completeness"}\NormalTok{)}
\end{Highlighting}
\end{Shaded}

\begin{figure}[H]

{\centering \pandocbounded{\includegraphics[keepaspectratio]{index_files/figure-pdf/structure-and-preview-1.pdf}}

}

\caption{The visual overview indicates that 75\% of variables are
continuous and 25\% are categorical, with no completely missing columns.
Approximately 92.7\% of rows are fully complete, and only 1.3\% of
observations contain missing values, suggesting minimal data loss prior
to imputation.}

\end{figure}%

\subsubsection{Missing Data Summary}\label{missing-data-summary}

\begin{Shaded}
\begin{Highlighting}[]
\CommentTok{\# Visualize missing data pattern}
\FunctionTok{plot\_missing}\NormalTok{(adult, }\AttributeTok{title =} \StringTok{"Missing Data Pattern (Adult Dataset)"}\NormalTok{)}
\end{Highlighting}
\end{Shaded}

\begin{figure}[H]

{\centering \pandocbounded{\includegraphics[keepaspectratio]{index_files/figure-pdf/missingness-visual-1.pdf}}

}

\caption{Missing data were minimal across analytic variables.
BMI-related fields (bmi, bmi\_c, bmi\_cat) showed \textasciitilde4.3\%
missingness, and diabetes\_dx showed \textasciitilde3.1\%. All
demographic and survey design variables were complete, indicating that
missingness was limited to health-related measures and appropriate for
multiple imputation.}

\end{figure}%

\begin{Shaded}
\begin{Highlighting}[]
\CommentTok{\# Summarize missingness for key analysis variables}

\NormalTok{miss\_tbl }\OtherTok{\textless{}{-}}\NormalTok{ tibble}\SpecialCharTok{::}\FunctionTok{tibble}\NormalTok{(}
\AttributeTok{Variable    =} \FunctionTok{c}\NormalTok{(}\StringTok{"bmi"}\NormalTok{, }\StringTok{"diabetes\_dx"}\NormalTok{),}
\AttributeTok{Missing\_n   =} \FunctionTok{c}\NormalTok{(}\FunctionTok{sum}\NormalTok{(}\FunctionTok{is.na}\NormalTok{(adult\_eda}\SpecialCharTok{$}\NormalTok{bmi)), }\FunctionTok{sum}\NormalTok{(}\FunctionTok{is.na}\NormalTok{(adult\_eda}\SpecialCharTok{$}\NormalTok{diabetes\_dx))),}
\AttributeTok{Missing\_pct =} \FunctionTok{round}\NormalTok{(}\FunctionTok{c}\NormalTok{(}\FunctionTok{mean}\NormalTok{(}\FunctionTok{is.na}\NormalTok{(adult\_eda}\SpecialCharTok{$}\NormalTok{bmi)), }\FunctionTok{mean}\NormalTok{(}\FunctionTok{is.na}\NormalTok{(adult\_eda}\SpecialCharTok{$}\NormalTok{diabetes\_dx))) }\SpecialCharTok{*} \DecValTok{100}\NormalTok{, }\DecValTok{1}\NormalTok{)}
\NormalTok{)}

\NormalTok{knitr}\SpecialCharTok{::}\FunctionTok{kable}\NormalTok{(}
\NormalTok{miss\_tbl,}
\AttributeTok{caption =} \StringTok{"Missingness for Key Analysis Variables."}
\NormalTok{)}
\end{Highlighting}
\end{Shaded}

\begin{longtable}[]{@{}lrr@{}}
\caption{Missingness for Key Analysis Variables.}\tabularnewline
\toprule\noalign{}
Variable & Missing\_n & Missing\_pct \\
\midrule\noalign{}
\endfirsthead
\toprule\noalign{}
Variable & Missing\_n & Missing\_pct \\
\midrule\noalign{}
\endhead
\bottomrule\noalign{}
\endlastfoot
bmi & 249 & 4.3 \\
diabetes\_dx & 177 & 3.1 \\
\end{longtable}

Overall missingness was low (\textasciitilde7.3\%). Gaps were
concentrated in \texttt{bmi} (n = 249) and \texttt{diabetes\_dx} (n =
177), while demographic and design variables were complete. This limited
pattern of missingness is consistent with a Missing At Random (MAR)
mechanism and likely reflects reduced participation in the physical
examination component among certain adults.

\subsubsection{Exploratory Data
Analysis}\label{exploratory-data-analysis}

Following the missing data assessment, exploratory analyses were
conducted to describe the adult analytic cohort and visualize
distributions across key demographic and health variables. The goal was
to examine univariate patterns and bivariate relationships relevant to
diabetes prevalence prior to modeling.

The adult analytic cohort was broadly representative of the U.S.
population, with a majority identifying as \texttt{Non-Hispanic\ White}.
\texttt{Age} and \texttt{BMI} distributions were right-skewed, with most
participants classified as overweight or obese. Visual exploration
revealed a clear positive association between \texttt{age},
\texttt{BMI}, and diabetes prevalence. \texttt{Non-Hispanic\ Black} and
\texttt{Hispanic} participants exhibited higher diabetes prevalence
compared with \texttt{Non-Hispanic\ Whites}.

Approximately 25\% of variables were categorical (e.g., \texttt{sex},
\texttt{race}, \texttt{diabetes\_dx}) and 75\% were continuous (e.g.,
\texttt{age}, \texttt{bmi}, \texttt{age\_c}, \texttt{bmi\_c}),
indicating that the dataset primarily comprised measured numeric values.
About 93\% of observations contained complete information across all
predictors and outcomes, reflecting high data quality.

Adult \texttt{age} ranged from 20 to 80 years, with peak representation
between 30 and 50 years and a slight right skew toward older ages.
\texttt{BMI} was concentrated in the overweight and obese ranges, and
\texttt{Female} participants were slightly overrepresented relative to
\texttt{Male} participants.

\begin{Shaded}
\begin{Highlighting}[]
\CommentTok{\# Age distribution (analytic adult)}
\FunctionTok{ggplot}\NormalTok{(adult, }\FunctionTok{aes}\NormalTok{(}\AttributeTok{x =}\NormalTok{ age)) }\SpecialCharTok{+}
  \FunctionTok{geom\_histogram}\NormalTok{(}\AttributeTok{binwidth =} \DecValTok{5}\NormalTok{, }\AttributeTok{color =} \StringTok{"white"}\NormalTok{) }\SpecialCharTok{+}
  \FunctionTok{labs}\NormalTok{(}\AttributeTok{title =} \StringTok{"Distribution of Age (≥20 years)"}\NormalTok{, }\AttributeTok{x =} \StringTok{"Age (years)"}\NormalTok{, }\AttributeTok{y =} \StringTok{"Count"}\NormalTok{) }\SpecialCharTok{+}
  \FunctionTok{theme\_minimal}\NormalTok{()}
\end{Highlighting}
\end{Shaded}

\begin{figure}[H]

{\centering \pandocbounded{\includegraphics[keepaspectratio]{index_files/figure-pdf/eda-age-distribution-1.pdf}}

}

\caption{Distribution of age among adults aged ≥20 years. The sample
spans 20--80 years, with peak representation between 30 and 50 years and
a gradual decline in older age groups, reflecting a balanced adult
cohort suitable for regression modeling.}

\end{figure}%

\begin{Shaded}
\begin{Highlighting}[]
\CommentTok{\# Diabetes outcome distribution}
\FunctionTok{ggplot}\NormalTok{(adult, }\FunctionTok{aes}\NormalTok{(}\AttributeTok{x =} \FunctionTok{factor}\NormalTok{(diabetes\_dx, }\AttributeTok{levels =} \FunctionTok{c}\NormalTok{(}\DecValTok{0}\NormalTok{,}\DecValTok{1}\NormalTok{), }\AttributeTok{labels =} \FunctionTok{c}\NormalTok{(}\StringTok{"No"}\NormalTok{,}\StringTok{"Yes"}\NormalTok{)))) }\SpecialCharTok{+}
  \FunctionTok{geom\_bar}\NormalTok{() }\SpecialCharTok{+}
  \FunctionTok{labs}\NormalTok{(}\AttributeTok{title =} \StringTok{"Diabetes Outcome Distribution (≥20 years)"}\NormalTok{, }\AttributeTok{x =} \StringTok{"Diabetes (No/Yes)"}\NormalTok{, }\AttributeTok{y =} \StringTok{"Count"}\NormalTok{) }\SpecialCharTok{+}
  \FunctionTok{theme\_minimal}\NormalTok{()}
\end{Highlighting}
\end{Shaded}

\begin{figure}[H]

{\centering \pandocbounded{\includegraphics[keepaspectratio]{index_files/figure-pdf/eda-diabetes-distribution-1.pdf}}

}

\caption{Distribution of diabetes outcomes among adults aged ≥20 years.
Most participants reported no diabetes diagnosis (\texttt{No}), while
approximately 11\% had diabetes (\texttt{Yes}) and 3\% had missing
responses, reflecting expected population prevalence and limited outcome
missingness.}

\end{figure}%

\begin{Shaded}
\begin{Highlighting}[]
\CommentTok{\# BMI category distribution}
\FunctionTok{ggplot}\NormalTok{(adult, }\FunctionTok{aes}\NormalTok{(}\AttributeTok{x =}\NormalTok{ bmi\_cat)) }\SpecialCharTok{+}
  \FunctionTok{geom\_bar}\NormalTok{(}\AttributeTok{color =} \StringTok{"white"}\NormalTok{, }\AttributeTok{fill =} \StringTok{"skyblue"}\NormalTok{) }\SpecialCharTok{+}
  \FunctionTok{labs}\NormalTok{(}\AttributeTok{title =} \StringTok{"Distribution of BMI Categories (≥20 years)"}\NormalTok{, }\AttributeTok{x =} \StringTok{"BMI Category"}\NormalTok{, }\AttributeTok{y =} \StringTok{"Count"}\NormalTok{) }\SpecialCharTok{+}
  \FunctionTok{theme\_minimal}\NormalTok{()}
\end{Highlighting}
\end{Shaded}

\begin{figure}[H]

{\centering \pandocbounded{\includegraphics[keepaspectratio]{index_files/figure-pdf/eda-bmi-distribution-1.pdf}}

}

\caption{Distribution of BMI categories among adults aged ≥20 years. The
majority of participants fell within the overweight (25--\textless30)
and obese (≥30) ranges, with fewer individuals classified as underweight
(\textless18.5). This distribution aligns with national trends in adult
body composition, supporting the dataset's representativeness for
metabolic health analyses.}

\end{figure}%

\begin{Shaded}
\begin{Highlighting}[]
\CommentTok{\# BMI by diabetes outcome (boxplot)}
\CommentTok{\# (You can’t use boxplot with categorical y, so revert to numeric BMI here)}
\FunctionTok{ggplot}\NormalTok{(adult, }\FunctionTok{aes}\NormalTok{(}\AttributeTok{x =} \FunctionTok{factor}\NormalTok{(diabetes\_dx, }\AttributeTok{levels =} \FunctionTok{c}\NormalTok{(}\DecValTok{0}\NormalTok{,}\DecValTok{1}\NormalTok{), }\AttributeTok{labels =} \FunctionTok{c}\NormalTok{(}\StringTok{"No"}\NormalTok{,}\StringTok{"Yes"}\NormalTok{)), }\AttributeTok{y =}\NormalTok{ bmi)) }\SpecialCharTok{+}
  \FunctionTok{geom\_boxplot}\NormalTok{(}\AttributeTok{fill =} \StringTok{"lightblue"}\NormalTok{) }\SpecialCharTok{+}
  \FunctionTok{labs}\NormalTok{(}\AttributeTok{title =} \StringTok{"BMI by Diabetes Diagnosis (≥20 years)"}\NormalTok{, }\AttributeTok{x =} \StringTok{"Diabetes (No/Yes)"}\NormalTok{, }\AttributeTok{y =} \StringTok{"BMI (numeric)"}\NormalTok{) }\SpecialCharTok{+}
  \FunctionTok{theme\_minimal}\NormalTok{()}
\end{Highlighting}
\end{Shaded}

\begin{figure}[H]

{\centering \pandocbounded{\includegraphics[keepaspectratio]{index_files/figure-pdf/eda-bmi-by-diabetes-outcome-1.pdf}}

}

\caption{Distribution of BMI by diabetes diagnosis among adults aged ≥20
years. Participants with diabetes (\texttt{Yes}) show a higher median
BMI and greater variability compared to those without diabetes
(\texttt{No}), supporting the established positive association between
obesity and diabetes risk.}

\end{figure}%

\begin{Shaded}
\begin{Highlighting}[]
\CommentTok{\# Diabetes by race (dodged bars)}
\FunctionTok{ggplot}\NormalTok{(adult, }\FunctionTok{aes}\NormalTok{(}\AttributeTok{x =}\NormalTok{ race, }\AttributeTok{fill =} \FunctionTok{factor}\NormalTok{(diabetes\_dx, }\AttributeTok{levels =} \FunctionTok{c}\NormalTok{(}\DecValTok{0}\NormalTok{,}\DecValTok{1}\NormalTok{), }\AttributeTok{labels =} \FunctionTok{c}\NormalTok{(}\StringTok{"No"}\NormalTok{,}\StringTok{"Yes"}\NormalTok{)))) }\SpecialCharTok{+}
  \FunctionTok{geom\_bar}\NormalTok{(}\AttributeTok{position =} \StringTok{"dodge"}\NormalTok{) }\SpecialCharTok{+}
  \FunctionTok{labs}\NormalTok{(}\AttributeTok{title =} \StringTok{"Diabetes Diagnosis by race/Ethnicity (≥20 years)"}\NormalTok{,}
       \AttributeTok{x =} \StringTok{"race/Ethnicity (race)"}\NormalTok{, }\AttributeTok{y =} \StringTok{"Count"}\NormalTok{, }\AttributeTok{fill =} \StringTok{"Diabetes"}\NormalTok{) }\SpecialCharTok{+}
  \FunctionTok{theme\_minimal}\NormalTok{() }\SpecialCharTok{+}
  \FunctionTok{theme}\NormalTok{(}\AttributeTok{axis.text.x =} \FunctionTok{element\_text}\NormalTok{(}\AttributeTok{angle =} \DecValTok{45}\NormalTok{, }\AttributeTok{hjust =} \DecValTok{1}\NormalTok{))}
\end{Highlighting}
\end{Shaded}

\begin{figure}[H]

{\centering \pandocbounded{\includegraphics[keepaspectratio]{index_files/figure-pdf/eda-diabetes-by-race-1.pdf}}

}

\caption{Diabetes diagnosis by race/ethnicity among adults aged ≥20
years. Non-Hispanic Black and Hispanic participants show higher
proportions of diabetes diagnoses compared with Non-Hispanic White
participants, reflecting known disparities in diabetes prevalence across
racial and ethnic groups.}

\end{figure}%

\subsection{Modeling Frameworks}\label{modeling-frameworks}

Three modeling frameworks were compared using identical
predictors---standardized age, BMI, sex, and race---and the binary
outcome \texttt{diabetes\_dx}:

\begin{enumerate}
\def\labelenumi{(\arabic{enumi})}
\item
  survey-weighted logistic regression to account for the NHANES complex
  sampling design,
\item
  multiple imputation (MICE) to handle missing BMI values, and
\item
  Bayesian logistic regression with weakly informative priors to
  quantify parameter uncertainty.
\end{enumerate}

\subsubsection{Survey-Weighted Logistic Regression (Design-Based
MLE)}\label{survey-weighted-logistic-regression-design-based-mle}

The NHANES 2013--2014 data use a complex, multistage probability design
involving strata (\texttt{SDMVSTRA}), primary sampling units (PSUs;
\texttt{SDMVPSU}), and examination weights (\texttt{WTMEC2YR}) to ensure
nationally representative estimates (\citeproc{ref-nchs2014}{National
Center for Health Statistics (NCHS) 2014}).

Estimates are population-weighted using NHANES survey design variables
(\texttt{WTMEC2YR}, \texttt{SDMVSTRA}, \texttt{SDMVPSU}). Odds ratios
are reported per one standard deviation (1 SD) increase in age and BMI,
with reference groups Male and White.

\begin{Shaded}
\begin{Highlighting}[]
\NormalTok{adult\_clean }\OtherTok{\textless{}{-}}\NormalTok{ adult }\SpecialCharTok{\%\textgreater{}\%}
\NormalTok{  dplyr}\SpecialCharTok{::}\FunctionTok{mutate}\NormalTok{(}
    \AttributeTok{sex   =}\NormalTok{ forcats}\SpecialCharTok{::}\FunctionTok{fct\_drop}\NormalTok{(sex),}
    \AttributeTok{race =}\NormalTok{ forcats}\SpecialCharTok{::}\FunctionTok{fct\_drop}\NormalTok{(race),}
    \AttributeTok{age\_c =} \FunctionTok{as.numeric}\NormalTok{(age\_c),}
    \AttributeTok{bmi\_c =} \FunctionTok{as.numeric}\NormalTok{(bmi\_c)}
\NormalTok{  ) }\SpecialCharTok{\%\textgreater{}\%}
\NormalTok{  dplyr}\SpecialCharTok{::}\FunctionTok{filter}\NormalTok{(}
    \SpecialCharTok{!}\FunctionTok{is.na}\NormalTok{(diabetes\_dx),}
    \SpecialCharTok{!}\FunctionTok{is.na}\NormalTok{(age\_c),}
    \SpecialCharTok{!}\FunctionTok{is.na}\NormalTok{(bmi\_c),}
    \SpecialCharTok{!}\FunctionTok{is.na}\NormalTok{(sex),}
    \SpecialCharTok{!}\FunctionTok{is.na}\NormalTok{(race)}
\NormalTok{  )}
\end{Highlighting}
\end{Shaded}

Below is a structure of the analytic dataset used for regression
modeling, showing variable names, types, and sample values (N = 5,349).

\begin{Shaded}
\begin{Highlighting}[]
\FunctionTok{str}\NormalTok{(adult\_clean[, }\FunctionTok{c}\NormalTok{(}\StringTok{"diabetes\_dx"}\NormalTok{,}\StringTok{"sex"}\NormalTok{,}\StringTok{"race"}\NormalTok{,}\StringTok{"age\_c"}\NormalTok{,}\StringTok{"bmi\_c"}\NormalTok{)])}
\end{Highlighting}
\end{Shaded}

\begin{verbatim}
'data.frame':   5349 obs. of  5 variables:
 $ diabetes_dx: num  1 1 1 0 0 0 0 0 0 1 ...
 $ sex        : Factor w/ 2 levels "Male","Female": 1 1 1 2 1 2 2 1 2 1 ...
 $ race       : Factor w/ 5 levels "NH White","Mexican American",..: 4 1 1 1 2 1 1 1 1 1 ...
 $ age_c      : num  1.132 0.278 1.303 1.36 0.392 ...
 $ bmi_c      : num  -0.3359 -0.0703 -0.0283 -1.3144 1.761 ...
\end{verbatim}

\begin{Shaded}
\begin{Highlighting}[]
\NormalTok{knitr}\SpecialCharTok{::}\FunctionTok{kable}\NormalTok{(}
  \FunctionTok{table}\NormalTok{(adult\_clean}\SpecialCharTok{$}\NormalTok{sex)}
\NormalTok{)}
\end{Highlighting}
\end{Shaded}

\begin{longtable}[]{@{}lr@{}}
\caption{Distribution of participants by sex (Male = 2,551; Female =
2,798) in the analytic cohort.}\tabularnewline
\toprule\noalign{}
Var1 & Freq \\
\midrule\noalign{}
\endfirsthead
\toprule\noalign{}
Var1 & Freq \\
\midrule\noalign{}
\endhead
\bottomrule\noalign{}
\endlastfoot
Male & 2551 \\
Female & 2798 \\
\end{longtable}

\begin{Shaded}
\begin{Highlighting}[]
\NormalTok{knitr}\SpecialCharTok{::}\FunctionTok{kable}\NormalTok{(}
  \FunctionTok{table}\NormalTok{(adult\_clean}\SpecialCharTok{$}\NormalTok{race)}
\NormalTok{)}
\end{Highlighting}
\end{Shaded}

\begin{longtable}[]{@{}lr@{}}
\caption{Race/ethnicity composition of the analytic cohort, with most
participants identifying as Non-Hispanic White (n = 2,293) and
Non-Hispanic Black (n = 1,101).}\tabularnewline
\toprule\noalign{}
Var1 & Freq \\
\midrule\noalign{}
\endfirsthead
\toprule\noalign{}
Var1 & Freq \\
\midrule\noalign{}
\endhead
\bottomrule\noalign{}
\endlastfoot
NH White & 2293 \\
Mexican American & 713 \\
Other Hispanic & 470 \\
NH Black & 1101 \\
Other/Multi & 772 \\
\end{longtable}

\begin{Shaded}
\begin{Highlighting}[]
\NormalTok{knitr}\SpecialCharTok{::}\FunctionTok{kable}\NormalTok{(}
  \FunctionTok{table}\NormalTok{(adult\_clean}\SpecialCharTok{$}\NormalTok{diabetes\_dx)}
\NormalTok{)}
\end{Highlighting}
\end{Shaded}

\begin{longtable}[]{@{}lr@{}}
\caption{Observed diabetes prevalence (binary outcome variable
\texttt{diabetes\_dx}), with 597 diagnosed cases (1 = Yes) and 4,752
non-diabetic participants (0 = No).}\tabularnewline
\toprule\noalign{}
Var1 & Freq \\
\midrule\noalign{}
\endfirsthead
\toprule\noalign{}
Var1 & Freq \\
\midrule\noalign{}
\endhead
\bottomrule\noalign{}
\endlastfoot
0 & 4752 \\
1 & 597 \\
\end{longtable}

\begin{Shaded}
\begin{Highlighting}[]
\FunctionTok{options}\NormalTok{(}\AttributeTok{survey.lonely.psu =} \StringTok{"adjust"}\NormalTok{)}

\NormalTok{nhanes\_design\_adult }\OtherTok{\textless{}{-}}\NormalTok{ survey}\SpecialCharTok{::}\FunctionTok{svydesign}\NormalTok{(}
  \AttributeTok{id      =} \SpecialCharTok{\textasciitilde{}}\NormalTok{SDMVPSU,}
  \AttributeTok{strata  =} \SpecialCharTok{\textasciitilde{}}\NormalTok{SDMVSTRA,}
  \AttributeTok{weights =} \SpecialCharTok{\textasciitilde{}}\NormalTok{WTMEC2YR,}
  \AttributeTok{nest    =} \ConstantTok{TRUE}\NormalTok{,}
  \AttributeTok{data    =}\NormalTok{ adult\_clean}
\NormalTok{)}

\NormalTok{svy\_fit }\OtherTok{\textless{}{-}}\NormalTok{ survey}\SpecialCharTok{::}\FunctionTok{svyglm}\NormalTok{(}
\NormalTok{  diabetes\_dx }\SpecialCharTok{\textasciitilde{}}\NormalTok{ age\_c }\SpecialCharTok{+}\NormalTok{ bmi\_c }\SpecialCharTok{+}\NormalTok{ sex }\SpecialCharTok{+}\NormalTok{ race,}
  \AttributeTok{design =}\NormalTok{ nhanes\_design\_adult,}
  \AttributeTok{family =} \FunctionTok{quasibinomial}\NormalTok{()}
\NormalTok{)}

\NormalTok{svy\_or }\OtherTok{\textless{}{-}}\NormalTok{ broom}\SpecialCharTok{::}\FunctionTok{tidy}\NormalTok{(svy\_fit, }\AttributeTok{conf.int =} \ConstantTok{TRUE}\NormalTok{) }\SpecialCharTok{\%\textgreater{}\%}
\NormalTok{  dplyr}\SpecialCharTok{::}\FunctionTok{mutate}\NormalTok{(}
    \AttributeTok{OR  =} \FunctionTok{exp}\NormalTok{(estimate),}
    \AttributeTok{LCL =} \FunctionTok{exp}\NormalTok{(conf.low),}
    \AttributeTok{UCL =} \FunctionTok{exp}\NormalTok{(conf.high)}
\NormalTok{  ) }\SpecialCharTok{\%\textgreater{}\%}
\NormalTok{  dplyr}\SpecialCharTok{::}\FunctionTok{select}\NormalTok{(term, OR, LCL, UCL, p.value) }\SpecialCharTok{\%\textgreater{}\%}
\NormalTok{  dplyr}\SpecialCharTok{::}\FunctionTok{filter}\NormalTok{(term }\SpecialCharTok{!=} \StringTok{"(Intercept)"}\NormalTok{)}
\end{Highlighting}
\end{Shaded}

\begin{Shaded}
\begin{Highlighting}[]
\NormalTok{knitr}\SpecialCharTok{::}\FunctionTok{kable}\NormalTok{(svy\_or)}
\end{Highlighting}
\end{Shaded}

\begin{longtable}[]{@{}lrrrr@{}}

\caption{\label{tbl-svylogit}Survey-weighted logistic regression: odds
ratios (OR) and 95\% confidence intervals for diabetes diagnosis among
adults (NHANES 2013--2014).}

\tabularnewline

\toprule\noalign{}
term & OR & LCL & UCL & p.value \\
\midrule\noalign{}
\endhead
\bottomrule\noalign{}
\endlastfoot
age\_c & 3.0292807 & 2.6967690 & 3.4027912 & 0.0000000 \\
bmi\_c & 1.8853571 & 1.6526296 & 2.1508579 & 0.0000039 \\
sexFemale & 0.5281132 & 0.4104905 & 0.6794397 & 0.0003857 \\
raceMexican American & 2.0358434 & 1.4850041 & 2.7910081 & 0.0008262 \\
raceOther Hispanic & 1.5915182 & 1.1664529 & 2.1714810 & 0.0087119 \\
raceNH Black & 1.6689718 & 1.1605895 & 2.4000450 & 0.0116773 \\
raceOther/Multi & 2.3270527 & 1.5451752 & 3.5045697 & 0.0014331 \\

\end{longtable}

\paragraph{Interpretation}\label{interpretation}

\texttt{age\_c} and \texttt{bmi\_c} are the strongest predictors of
diabetes in the NHANES 2013--2014 adult cohort, with each 1 SD increase
in age nearly tripling the odds of diabetes and higher BMI substantially
elevating risk. Males show significantly lower odds of diabetes than
females, consistent with established sex differences in metabolic
outcomes. Racial and ethnic disparities are evident, with Mexican
American, Other Hispanic, Non-Hispanic Black, and Other/Multi-racial
adults all showing significantly higher odds of diabetes compared to
Non-Hispanic Whites. All predictors were statistically significant (p
\textless{} 0.05), indicating robust associations across demographic and
health characteristics.

\subsubsection{Multiple Imputation by Chained
Equations}\label{multiple-imputation-by-chained-equations}

Multiple Imputation by Chained Equations (\texttt{MICE}) was implemented
as a principled approach for handling missing data
(\citeproc{ref-vanbuuren2011}{Stef van Buuren and Groothuis-Oudshoorn
2011}; \citeproc{ref-vanbuuren2012}{S. van Buuren 2012}). \texttt{MICE}
iteratively imputes each incomplete variable using regression models
based on other variables in the dataset, generating multiple completed
datasets that incorporate uncertainty from the imputation process.
Estimates are subsequently pooled across imputations using Rubin's rules
to obtain final parameter estimates and confidence intervals.

As an alternative to full Bayesian joint modeling, \texttt{MICE}
provides an efficient and flexible framework for managing missing data
through chained regression equations. For large sample sizes
(\texttt{n\ ≥\ 400}), even with substantial missingness (up to 75\%) in
a single variable, \texttt{MICE} remains robust to non-normality---such
as skewed, multimodal, or heavy-tailed distributions---without
materially affecting mean structure estimation performance
(\citeproc{ref-vanbuuren2012}{S. van Buuren 2012}).

In this study, continuous variables were imputed using regression-based
methods: \texttt{age} via normal linear regression (\texttt{norm}) and
\texttt{BMI} via predictive mean matching (\texttt{pmm}) to better
preserve the empirical BMI distribution. Categorical variables
(\texttt{sex} and \texttt{race}) were imputed using logistic and
polytomous regression models, respectively. Diabetes status
(\texttt{diabetes\_dx}) was treated as an outcome variable and was
\textbf{not} imputed. Twenty imputations were generated to minimize
Monte Carlo error and ensure stable variance estimation.

\paragraph{Convergence and Data
Stability}\label{convergence-and-data-stability}

The chained equation process showed stable convergence across
iterations, confirming reliable estimation of missing \texttt{BMI} (and,
where present, \texttt{age}) values. After applying \texttt{MICE}, the
final imputed dataset included \textbf{n = 5,592 adults} with all key
predictors completed.

\begin{Shaded}
\begin{Highlighting}[]
\NormalTok{adult\_imp1 }\OtherTok{\textless{}{-}}\NormalTok{ mice}\SpecialCharTok{::}\FunctionTok{complete}\NormalTok{(imp, }\DecValTok{1}\NormalTok{) }\SpecialCharTok{\%\textgreater{}\%}
\NormalTok{  dplyr}\SpecialCharTok{::}\FunctionTok{mutate}\NormalTok{(}
    \AttributeTok{age\_c  =} \FunctionTok{as.numeric}\NormalTok{(}\FunctionTok{scale}\NormalTok{(age)),}
    \AttributeTok{bmi\_c  =} \FunctionTok{as.numeric}\NormalTok{(}\FunctionTok{scale}\NormalTok{(bmi)),}
    \AttributeTok{wt\_norm =}\NormalTok{ WTMEC2YR }\SpecialCharTok{/} \FunctionTok{mean}\NormalTok{(WTMEC2YR, }\AttributeTok{na.rm =} \ConstantTok{TRUE}\NormalTok{),}
    \AttributeTok{race =}\NormalTok{ forcats}\SpecialCharTok{::}\FunctionTok{fct\_relevel}\NormalTok{(race, }\StringTok{"NH White"}\NormalTok{),}
    \AttributeTok{sex  =}\NormalTok{ forcats}\SpecialCharTok{::}\FunctionTok{fct\_relevel}\NormalTok{(sex,  }\StringTok{"Male"}\NormalTok{)}
\NormalTok{  ) }\SpecialCharTok{\%\textgreater{}\%}
\NormalTok{  dplyr}\SpecialCharTok{::}\FunctionTok{filter}\NormalTok{(}
    \SpecialCharTok{!}\FunctionTok{is.na}\NormalTok{(diabetes\_dx),}
    \SpecialCharTok{!}\FunctionTok{is.na}\NormalTok{(age\_c),}
    \SpecialCharTok{!}\FunctionTok{is.na}\NormalTok{(bmi\_c),}
    \SpecialCharTok{!}\FunctionTok{is.na}\NormalTok{(sex),}
    \SpecialCharTok{!}\FunctionTok{is.na}\NormalTok{(race)}
\NormalTok{  ) }\SpecialCharTok{\%\textgreater{}\%}
  \FunctionTok{droplevels}\NormalTok{()}

\FunctionTok{glimpse}\NormalTok{(adult\_imp1)}
\end{Highlighting}
\end{Shaded}

\begin{verbatim}
Rows: 5,592
Columns: 11
$ diabetes_dx <dbl> 1, 1, 1, 0, 0, 0, 0, 0, 0, 0, 1, 0, 0, 0, 0, 0, 0, 0, 0, 0~
$ age         <dbl> 69, 54, 72, 73, 56, 61, 42, 56, 65, 26, 76, 33, 32, 38, 50~
$ bmi         <dbl> 26.7, 28.6, 28.9, 19.7, 41.7, 35.7, 23.6, 26.5, 22.0, 20.3~
$ sex         <fct> Male, Male, Male, Female, Male, Female, Male, Female, Male~
$ race        <fct> NH Black, NH White, NH White, NH White, Mexican American, ~
$ WTMEC2YR    <dbl> 13481.04, 24471.77, 57193.29, 65541.87, 25344.99, 61758.65~
$ SDMVPSU     <dbl> 1, 1, 1, 2, 1, 1, 2, 1, 2, 2, 1, 2, 2, 2, 2, 1, 1, 1, 2, 2~
$ SDMVSTRA    <dbl> 112, 108, 109, 116, 111, 114, 106, 112, 112, 113, 116, 114~
$ age_c       <dbl> 1.13241831, 0.27835981, 1.30323001, 1.36016725, 0.39223428~
$ bmi_c       <dbl> -0.33319172, -0.06755778, -0.02561558, -1.31184309, 1.7639~
$ wt_norm     <dbl> 0.3393916, 0.6160884, 1.4398681, 1.6500477, 0.6380722, 1.5~
\end{verbatim}

\paragraph{Descriptive Results (Imputed
Dataset)}\label{descriptive-results-imputed-dataset}

After imputation, the analytic dataset contained approximately
\textbf{5,500--5,600 adults}. The mean \texttt{age} was around
\textbf{49 years} (SD ≈ 17), and the mean \texttt{BMI} was approximately
\textbf{29} (SD ≈ 7). \texttt{Female} participants represented about
\textbf{52\%} of the sample, and the majority identified as
\texttt{Non-Hispanic\ White} (\textasciitilde43\%). The estimated
diabetes prevalence was \textbf{\textasciitilde11\%}, consistent with
population-level NHANES benchmarks.

\begin{Shaded}
\begin{Highlighting}[]
\NormalTok{correlation\_matrix }\OtherTok{\textless{}{-}} \FunctionTok{cor}\NormalTok{(adult\_imp1[, }\FunctionTok{c}\NormalTok{(}\StringTok{"diabetes\_dx"}\NormalTok{, }\StringTok{"age"}\NormalTok{, }\StringTok{"bmi"}\NormalTok{)], }\AttributeTok{use =} \StringTok{"complete.obs"}\NormalTok{, }\AttributeTok{method =} \StringTok{"pearson"}\NormalTok{)}
\NormalTok{correlation\_melted }\OtherTok{\textless{}{-}} \FunctionTok{melt}\NormalTok{(correlation\_matrix)}

\FunctionTok{ggplot}\NormalTok{(correlation\_melted, }\FunctionTok{aes}\NormalTok{(Var1, Var2, }\AttributeTok{fill =}\NormalTok{ value)) }\SpecialCharTok{+}
\FunctionTok{geom\_tile}\NormalTok{(}\AttributeTok{color =} \StringTok{"white"}\NormalTok{) }\SpecialCharTok{+}
\FunctionTok{scale\_fill\_gradient2}\NormalTok{(}\AttributeTok{low =} \StringTok{"blue"}\NormalTok{, }\AttributeTok{high =} \StringTok{"red"}\NormalTok{, }\AttributeTok{mid =} \StringTok{"white"}\NormalTok{, }\AttributeTok{midpoint =} \DecValTok{0}\NormalTok{, }\AttributeTok{name =} \StringTok{"Correlation"}\NormalTok{) }\SpecialCharTok{+}
\FunctionTok{theme\_minimal}\NormalTok{() }\SpecialCharTok{+}
\FunctionTok{theme}\NormalTok{(}\AttributeTok{axis.text.x =} \FunctionTok{element\_text}\NormalTok{(}\AttributeTok{angle =} \DecValTok{45}\NormalTok{, }\AttributeTok{hjust =} \DecValTok{1}\NormalTok{)) }\SpecialCharTok{+}
\FunctionTok{labs}\NormalTok{(}\AttributeTok{title =} \StringTok{"Correlation Heatmap: Diabetes, Age, and BMI"}\NormalTok{)}
\end{Highlighting}
\end{Shaded}

\begin{figure}[H]

{\centering \pandocbounded{\includegraphics[keepaspectratio]{index_files/figure-pdf/corr-heatmap-1.pdf}}

}

\caption{Correlation heatmap showing positive associations among
\texttt{diabetes\_dx}, \texttt{age}, and \texttt{BMI}. Both \texttt{age}
and \texttt{BMI} exhibit moderate positive correlations with diabetes
diagnosis, consistent with known metabolic risk trends in the NHANES
adult population.}

\end{figure}%

\begin{Shaded}
\begin{Highlighting}[]
\FunctionTok{ggplot}\NormalTok{(adult\_imp1, }\FunctionTok{aes}\NormalTok{(}\AttributeTok{x =} \FunctionTok{factor}\NormalTok{(diabetes\_dx))) }\SpecialCharTok{+}
\FunctionTok{geom\_bar}\NormalTok{(}\AttributeTok{fill =} \StringTok{"steelblue"}\NormalTok{) }\SpecialCharTok{+}
\FunctionTok{labs}\NormalTok{(}\AttributeTok{title =} \StringTok{"Diabetes Diagnosis Distribution"}\NormalTok{, }\AttributeTok{x =} \StringTok{"Diabetes (0 = No, 1 = Yes)"}\NormalTok{, }\AttributeTok{y =} \StringTok{"Count"}\NormalTok{) }\SpecialCharTok{+}
\FunctionTok{theme\_minimal}\NormalTok{()}
\end{Highlighting}
\end{Shaded}

\begin{figure}[H]

{\centering \pandocbounded{\includegraphics[keepaspectratio]{index_files/figure-pdf/diabetes-dist-1.pdf}}

}

\caption{Distribution of diabetes diagnosis among adults (age ≥ 20
years). The majority of participants (≈ 89\%) reported no diabetes
diagnosis (\texttt{0}), while about 11\% reported a positive diagnosis
(\texttt{1}), consistent with NHANES population prevalence.}

\end{figure}%

\begin{Shaded}
\begin{Highlighting}[]
\FunctionTok{ggplot}\NormalTok{(adult\_imp1, }\FunctionTok{aes}\NormalTok{(}\AttributeTok{x =} \FunctionTok{factor}\NormalTok{(diabetes\_dx), }\AttributeTok{y =}\NormalTok{ bmi, }\AttributeTok{fill =} \FunctionTok{factor}\NormalTok{(diabetes\_dx))) }\SpecialCharTok{+}
\FunctionTok{geom\_boxplot}\NormalTok{(}\AttributeTok{alpha =} \FloatTok{0.7}\NormalTok{) }\SpecialCharTok{+}
\FunctionTok{scale\_x\_discrete}\NormalTok{(}\AttributeTok{labels =} \FunctionTok{c}\NormalTok{(}\StringTok{"0"} \OtherTok{=} \StringTok{"No Diabetes"}\NormalTok{, }\StringTok{"1"} \OtherTok{=} \StringTok{"Diabetes"}\NormalTok{)) }\SpecialCharTok{+}
\FunctionTok{labs}\NormalTok{(}\AttributeTok{x =} \StringTok{"Diabetes Diagnosis"}\NormalTok{, }\AttributeTok{y =} \StringTok{"BMI"}\NormalTok{, }\AttributeTok{title =} \StringTok{"BMI Distribution by Diabetes Status"}\NormalTok{) }\SpecialCharTok{+}
\FunctionTok{theme\_minimal}\NormalTok{() }\SpecialCharTok{+}
\FunctionTok{theme}\NormalTok{(}\AttributeTok{legend.position =} \StringTok{"none"}\NormalTok{)}
\end{Highlighting}
\end{Shaded}

\begin{figure}[H]

{\centering \pandocbounded{\includegraphics[keepaspectratio]{index_files/figure-pdf/bmi-diabetes-box-1.pdf}}

}

\caption{Boxplot of BMI by diabetes status (\texttt{No} vs
\texttt{Diabetes}). This descriptive plot compares BMI distributions
between groups, highlighting higher median BMI, greater spread, and more
outliers in the diabetes group. It is used in the EDA to summarize group
differences before modeling.}

\end{figure}%

The boxplot in Figure @ref(fig-bmi-diabetes-box) is
\textbf{descriptive}: it summarizes the median, spread, and outliers in
BMI for participants with and without diabetes. The visibly higher
median and wider spread in the diabetes group reinforce the positive
association between excess adiposity and diabetes risk seen in later
regression models.

\begin{Shaded}
\begin{Highlighting}[]
\FunctionTok{ggplot}\NormalTok{(adult\_imp1, }\FunctionTok{aes}\NormalTok{(}\AttributeTok{x =}\NormalTok{ bmi, }\AttributeTok{y =}\NormalTok{ diabetes\_dx)) }\SpecialCharTok{+}
\FunctionTok{geom\_point}\NormalTok{(}\AttributeTok{alpha =} \FloatTok{0.2}\NormalTok{, }\AttributeTok{position =} \FunctionTok{position\_jitter}\NormalTok{(}\AttributeTok{height =} \FloatTok{0.02}\NormalTok{)) }\SpecialCharTok{+}
\FunctionTok{geom\_smooth}\NormalTok{(}\AttributeTok{method =} \StringTok{"glm"}\NormalTok{, }\AttributeTok{method.args =} \FunctionTok{list}\NormalTok{(}\AttributeTok{family =} \StringTok{"binomial"}\NormalTok{), }\AttributeTok{se =} \ConstantTok{TRUE}\NormalTok{, }\AttributeTok{color =} \StringTok{"blue"}\NormalTok{) }\SpecialCharTok{+}
\FunctionTok{labs}\NormalTok{(}\AttributeTok{x =} \StringTok{"BMI"}\NormalTok{, }\AttributeTok{y =} \StringTok{"Probability of Diabetes"}\NormalTok{, }\AttributeTok{title =} \StringTok{"Predicted Probability of Diabetes vs BMI"}\NormalTok{) }\SpecialCharTok{+}
\FunctionTok{theme\_minimal}\NormalTok{()}
\end{Highlighting}
\end{Shaded}

\begin{figure}[H]

{\centering \pandocbounded{\includegraphics[keepaspectratio]{index_files/figure-pdf/pred-bmi-1.pdf}}

}

\caption{Predicted probability of diabetes as a function of BMI from a
logistic regression model. This inferential plot visualizes the fitted
relationship between BMI and diabetes risk: the smooth curve shows the
modeled probability of diabetes, and the shaded band reflects
uncertainty (95\% confidence interval) around the fit.}

\end{figure}%

In contrast to the descriptive boxplot, Figure @ref(fig-pred-bmi) is
\textbf{inferential}: it displays the regression-based fitted
probability of diabetes across the BMI continuum. The non-linear,
increasing curve and its confidence band summarize how modeled diabetes
risk escalates with higher BMI.

\begin{Shaded}
\begin{Highlighting}[]
\NormalTok{miss\_age  }\OtherTok{\textless{}{-}} \FunctionTok{sum}\NormalTok{(}\FunctionTok{is.na}\NormalTok{(mi\_dat}\SpecialCharTok{$}\NormalTok{age))}
\NormalTok{miss\_bmiN }\OtherTok{\textless{}{-}} \FunctionTok{sum}\NormalTok{(}\FunctionTok{is.na}\NormalTok{(mi\_dat}\SpecialCharTok{$}\NormalTok{bmi))}

\NormalTok{mi\_caption }\OtherTok{\textless{}{-}} \ControlFlowTok{if}\NormalTok{ (miss\_age }\SpecialCharTok{\textgreater{}} \DecValTok{0} \SpecialCharTok{\&\&}\NormalTok{ miss\_bmiN }\SpecialCharTok{\textgreater{}} \DecValTok{0}\NormalTok{) \{}
\StringTok{"Multiple Imputation (MICE): pooled odds ratios (OR) and 95\% confidence intervals after imputing missing age (normal) and BMI (PMM) (m = 20); diabetes status was not imputed."}
\NormalTok{\} }\ControlFlowTok{else} \ControlFlowTok{if}\NormalTok{ (miss\_bmiN }\SpecialCharTok{\textgreater{}} \DecValTok{0}\NormalTok{) \{}
\StringTok{"Multiple Imputation (MICE): pooled odds ratios (OR) and 95\% confidence intervals after imputing missing BMI (PMM) (m = 20); diabetes status was not imputed."}
\NormalTok{\} }\ControlFlowTok{else} \ControlFlowTok{if}\NormalTok{ (miss\_age }\SpecialCharTok{\textgreater{}} \DecValTok{0}\NormalTok{) \{}
\StringTok{"Multiple Imputation (MICE): pooled odds ratios (OR) and 95\% confidence intervals after imputing missing age (normal) (m = 20); diabetes status was not imputed."}
\NormalTok{\} }\ControlFlowTok{else}\NormalTok{ \{}
\StringTok{"Multiple Imputation (MICE): pooled odds ratios (OR) and 95\% confidence intervals (no variables required imputation); diabetes status was not imputed."}
\NormalTok{\}}
\NormalTok{mi\_caption }\OtherTok{\textless{}{-}} \FunctionTok{paste0}\NormalTok{(mi\_caption, }\StringTok{" Odds ratios are per 1 SD for age and BMI."}\NormalTok{)}

\NormalTok{knitr}\SpecialCharTok{::}\FunctionTok{kable}\NormalTok{(mi\_or, }\AttributeTok{caption =}\NormalTok{ mi\_caption)}
\end{Highlighting}
\end{Shaded}

\begin{longtable}[]{@{}
  >{\raggedright\arraybackslash}p{(\linewidth - 20\tabcolsep) * \real{0.0265}}
  >{\raggedright\arraybackslash}p{(\linewidth - 20\tabcolsep) * \real{0.1858}}
  >{\raggedleft\arraybackslash}p{(\linewidth - 20\tabcolsep) * \real{0.0885}}
  >{\raggedleft\arraybackslash}p{(\linewidth - 20\tabcolsep) * \real{0.0885}}
  >{\raggedleft\arraybackslash}p{(\linewidth - 20\tabcolsep) * \real{0.0885}}
  >{\raggedleft\arraybackslash}p{(\linewidth - 20\tabcolsep) * \real{0.0796}}
  >{\raggedleft\arraybackslash}p{(\linewidth - 20\tabcolsep) * \real{0.0885}}
  >{\raggedleft\arraybackslash}p{(\linewidth - 20\tabcolsep) * \real{0.0885}}
  >{\raggedleft\arraybackslash}p{(\linewidth - 20\tabcolsep) * \real{0.0885}}
  >{\raggedleft\arraybackslash}p{(\linewidth - 20\tabcolsep) * \real{0.0885}}
  >{\raggedleft\arraybackslash}p{(\linewidth - 20\tabcolsep) * \real{0.0885}}@{}}

\caption{\label{tbl-mice}Multiple Imputation (MICE): pooled odds ratios
(OR) and 95\% confidence intervals after imputing missing BMI (PMM) (m =
20); diabetes status was not imputed. Odds ratios are per 1 SD for age
and BMI.}

\tabularnewline

\toprule\noalign{}
\begin{minipage}[b]{\linewidth}\raggedright
\end{minipage} & \begin{minipage}[b]{\linewidth}\raggedright
term
\end{minipage} & \begin{minipage}[b]{\linewidth}\raggedleft
OR
\end{minipage} & \begin{minipage}[b]{\linewidth}\raggedleft
std.error
\end{minipage} & \begin{minipage}[b]{\linewidth}\raggedleft
statistic
\end{minipage} & \begin{minipage}[b]{\linewidth}\raggedleft
df
\end{minipage} & \begin{minipage}[b]{\linewidth}\raggedleft
p.value
\end{minipage} & \begin{minipage}[b]{\linewidth}\raggedleft
LCL
\end{minipage} & \begin{minipage}[b]{\linewidth}\raggedleft
UCL
\end{minipage} & \begin{minipage}[b]{\linewidth}\raggedleft
conf.low
\end{minipage} & \begin{minipage}[b]{\linewidth}\raggedleft
conf.high
\end{minipage} \\
\midrule\noalign{}
\endhead
\bottomrule\noalign{}
\endlastfoot
2 & age\_c & 2.9038183 & 0.0559473 & 19.054108 & 5520.446 & 0.0000000 &
2.6021752 & 3.2404277 & 2.6021752 & 3.2404277 \\
3 & bmi\_c & 1.7278084 & 0.0447339 & 12.224604 & 5148.557 & 0.0000000 &
1.5827382 & 1.8861754 & 1.5827382 & 1.8861754 \\
4 & sexFemale & 0.5391132 & 0.0937913 & -6.587282 & 5551.660 & 0.0000000
& 0.4485669 & 0.6479368 & 0.4485669 & 0.6479368 \\
5 & raceMexican American & 2.4296216 & 0.1375046 & 6.456041 & 5472.583 &
0.0000000 & 1.8555327 & 3.1813298 & 1.8555327 & 3.1813298 \\
6 & raceOther Hispanic & 1.7518320 & 0.1748554 & 3.206433 & 5573.987 &
0.0013515 & 1.2434346 & 2.4680953 & 1.2434346 & 2.4680953 \\
7 & raceNH Black & 1.9757793 & 0.1198118 & 5.683602 & 5576.734 &
0.0000000 & 1.5621842 & 2.4988753 & 1.5621842 & 2.4988753 \\
8 & raceOther/Multi & 2.1120110 & 0.1530066 & 4.886328 & 4749.963 &
0.0000011 & 1.5646727 & 2.8508138 & 1.5646727 & 2.8508138 \\

\end{longtable}

\paragraph{Interpretation}\label{interpretation-1}

\begin{itemize}
\tightlist
\item
  \texttt{Age} and \texttt{BMI} are strong positive predictors of
  diabetes; each 1 SD increase substantially increases the odds of
  diagnosis.\\
\item
  \texttt{Sex:} Females exhibit significantly lower odds of diabetes
  compared to males.\\
\item
  \texttt{Race/Ethnicity:} All non-White racial and ethnic groups
  demonstrate higher odds of diabetes compared to Non-Hispanic Whites,
  underscoring persistent disparities in diabetes risk.\\
\item
  \textbf{Model Significance:} All predictors are statistically
  significant (\emph{p} \textless{} 0.05).\\
\item
  \textbf{Model Robustness:} Results are consistent with those from the
  survey-weighted model, confirming stability across imputation and
  weighting approaches.
\end{itemize}

\subsubsection{Bayesian Logistic
Regression}\label{bayesian-logistic-regression-1}

Bayesian logistic regression was used to quantify parameter uncertainty
and compare posterior estimates with the survey-weighted and MICE
models. Weakly informative priors were applied to regularize estimates
while preserving flexibility in inference.

\textbf{Model Specifications:} - \textbf{Family:} Bernoulli with logit
link\\
- \textbf{Data:} \texttt{adult\_imp1} (N = 5,592)\\
- \textbf{Chains:} 4 (2,000 iterations each; 1,000 warmup)\\
- \textbf{Adaptation delta:} 0.95\\
- \textbf{Weights:} Normalized NHANES examination weights
(\texttt{wt\_norm}, mean ≈ 1.00, SD ≈ 0.79)\\
- \textbf{Predictors:} Standardized \texttt{age}, \texttt{BMI},
\texttt{sex}, and \texttt{race}

\paragraph{Define Model and Priors}\label{define-model-and-priors}

\begin{Shaded}
\begin{Highlighting}[]
\NormalTok{fml\_bayes }\OtherTok{\textless{}{-}}\NormalTok{ diabetes\_dx }\SpecialCharTok{|} \FunctionTok{weights}\NormalTok{(wt\_norm) }\SpecialCharTok{\textasciitilde{}}\NormalTok{ age\_c }\SpecialCharTok{+}\NormalTok{ bmi\_c }\SpecialCharTok{+}\NormalTok{ sex }\SpecialCharTok{+}\NormalTok{ race}

\NormalTok{priors }\OtherTok{\textless{}{-}} \FunctionTok{c}\NormalTok{(}
\NormalTok{  brms}\SpecialCharTok{::}\FunctionTok{set\_prior}\NormalTok{(}\StringTok{"normal(0, 2.5)"}\NormalTok{, }\AttributeTok{class =} \StringTok{"b"}\NormalTok{),}
\NormalTok{  brms}\SpecialCharTok{::}\FunctionTok{set\_prior}\NormalTok{(}\StringTok{"student\_t(3, 0, 10)"}\NormalTok{, }\AttributeTok{class =} \StringTok{"Intercept"}\NormalTok{)}
\NormalTok{)}
\end{Highlighting}
\end{Shaded}

\begin{Shaded}
\begin{Highlighting}[]
\NormalTok{adult\_long }\OtherTok{\textless{}{-}}\NormalTok{ adult\_imp1 }\SpecialCharTok{\%\textgreater{}\%}
\NormalTok{dplyr}\SpecialCharTok{::}\FunctionTok{select}\NormalTok{(bmi\_c, age\_c) }\SpecialCharTok{\%\textgreater{}\%}
\NormalTok{tidyr}\SpecialCharTok{::}\FunctionTok{pivot\_longer}\NormalTok{(}
\AttributeTok{cols =}\NormalTok{ dplyr}\SpecialCharTok{::}\FunctionTok{everything}\NormalTok{(),}
\AttributeTok{names\_to =} \StringTok{"Coefficient"}\NormalTok{,}
\AttributeTok{values\_to =} \StringTok{"Value"}
\NormalTok{)}

\NormalTok{ggplot2}\SpecialCharTok{::}\FunctionTok{ggplot}\NormalTok{(adult\_long, ggplot2}\SpecialCharTok{::}\FunctionTok{aes}\NormalTok{(}\AttributeTok{x =}\NormalTok{ Value, }\AttributeTok{fill =}\NormalTok{ Coefficient)) }\SpecialCharTok{+}
\NormalTok{ggplot2}\SpecialCharTok{::}\FunctionTok{geom\_density}\NormalTok{(}\AttributeTok{alpha =} \FloatTok{0.5}\NormalTok{) }\SpecialCharTok{+}
\NormalTok{ggplot2}\SpecialCharTok{::}\FunctionTok{theme\_minimal}\NormalTok{() }\SpecialCharTok{+}
\NormalTok{ggplot2}\SpecialCharTok{::}\FunctionTok{labs}\NormalTok{(}
\AttributeTok{title =} \StringTok{"Distributions for Standardized Age and BMI (adult\_imp1)"}\NormalTok{,}
\AttributeTok{x =} \StringTok{"Standardized value (z{-}score)"}\NormalTok{,}
\AttributeTok{y =} \StringTok{"Density"}\NormalTok{,}
\AttributeTok{fill =} \StringTok{"Coefficient"}
\NormalTok{)}
\end{Highlighting}
\end{Shaded}

\begin{figure}[H]

{\centering \pandocbounded{\includegraphics[keepaspectratio]{index_files/figure-pdf/dist-adult-std-age-bmi-1.pdf}}

}

\caption{Distribution of standardized age (\texttt{age\_c}) and BMI
(\texttt{bmi\_c}) in the imputed dataset (\texttt{adult\_imp1}). Both
variables were mean-centered and scaled (z-scores) for inclusion in
regression models. The overlapping density curves indicate approximate
normality and comparable variance, supporting suitability for
standardized coefficient estimation.}

\end{figure}%

\begin{Shaded}
\begin{Highlighting}[]
\NormalTok{prior\_draws }\OtherTok{\textless{}{-}}\NormalTok{ tibble}\SpecialCharTok{::}\FunctionTok{tibble}\NormalTok{(}
\AttributeTok{term =} \FunctionTok{rep}\NormalTok{(}\FunctionTok{c}\NormalTok{(}\StringTok{"Age (per 1 SD)"}\NormalTok{, }\StringTok{"BMI (per 1 SD)"}\NormalTok{), }\AttributeTok{each =} \DecValTok{4000}\NormalTok{),}
\AttributeTok{value =} \FunctionTok{c}\NormalTok{(}
\NormalTok{stats}\SpecialCharTok{::}\FunctionTok{rnorm}\NormalTok{(}\DecValTok{4000}\NormalTok{, }\AttributeTok{mean =} \DecValTok{0}\NormalTok{, }\AttributeTok{sd =} \FloatTok{2.5}\NormalTok{),}
\NormalTok{stats}\SpecialCharTok{::}\FunctionTok{rnorm}\NormalTok{(}\DecValTok{4000}\NormalTok{, }\AttributeTok{mean =} \DecValTok{0}\NormalTok{, }\AttributeTok{sd =} \FloatTok{2.5}\NormalTok{)}
\NormalTok{)}
\NormalTok{)}

\NormalTok{ggplot2}\SpecialCharTok{::}\FunctionTok{ggplot}\NormalTok{(prior\_draws, ggplot2}\SpecialCharTok{::}\FunctionTok{aes}\NormalTok{(}\AttributeTok{x =}\NormalTok{ value, }\AttributeTok{fill =}\NormalTok{ term)) }\SpecialCharTok{+}
\NormalTok{ggplot2}\SpecialCharTok{::}\FunctionTok{geom\_density}\NormalTok{(}\AttributeTok{alpha =} \FloatTok{0.5}\NormalTok{) }\SpecialCharTok{+}
\NormalTok{ggplot2}\SpecialCharTok{::}\FunctionTok{theme\_minimal}\NormalTok{() }\SpecialCharTok{+}
\NormalTok{ggplot2}\SpecialCharTok{::}\FunctionTok{labs}\NormalTok{(}
\AttributeTok{title =} \StringTok{"Prior Distributions for Age and BMI Coefficients"}\NormalTok{,}
\AttributeTok{x =} \StringTok{"Coefficient value"}\NormalTok{,}
\AttributeTok{y =} \StringTok{"Density"}\NormalTok{,}
\AttributeTok{fill =} \ConstantTok{NULL}
\NormalTok{)}
\end{Highlighting}
\end{Shaded}

\begin{figure}[H]

{\centering \pandocbounded{\includegraphics[keepaspectratio]{index_files/figure-pdf/prior-age-bmi-1.pdf}}

}

\caption{Prior distributions for standardized age and BMI coefficients,
assuming Normal(0, 2.5) priors. These weakly informative priors
constrain extreme coefficient values while allowing flexibility in
posterior estimation, ensuring regularization without strong bias.}

\end{figure}%

\paragraph{Fit the Model}\label{fit-the-model}

\begin{Shaded}
\begin{Highlighting}[]
\NormalTok{priors }\OtherTok{\textless{}{-}} \FunctionTok{c}\NormalTok{(}
\NormalTok{  brms}\SpecialCharTok{::}\FunctionTok{set\_prior}\NormalTok{(}\StringTok{"normal(0, 2.5)"}\NormalTok{, }\AttributeTok{class =} \StringTok{"b"}\NormalTok{),}
\NormalTok{  brms}\SpecialCharTok{::}\FunctionTok{set\_prior}\NormalTok{(}\StringTok{"student\_t(3, 0, 10)"}\NormalTok{, }\AttributeTok{class =} \StringTok{"Intercept"}\NormalTok{)}
\NormalTok{)}

\NormalTok{bayes\_fit }\OtherTok{\textless{}{-}}\NormalTok{ brms}\SpecialCharTok{::}\FunctionTok{brm}\NormalTok{(}
  \AttributeTok{formula =}\NormalTok{ diabetes\_dx }\SpecialCharTok{|} \FunctionTok{weights}\NormalTok{(wt\_norm) }\SpecialCharTok{\textasciitilde{}}\NormalTok{ age\_c }\SpecialCharTok{+}\NormalTok{ bmi\_c }\SpecialCharTok{+}\NormalTok{ sex }\SpecialCharTok{+}\NormalTok{ race,}
  \AttributeTok{data    =}\NormalTok{ adult\_imp1,}
  \AttributeTok{family  =} \FunctionTok{bernoulli}\NormalTok{(}\AttributeTok{link =} \StringTok{"logit"}\NormalTok{),}
  \AttributeTok{prior   =}\NormalTok{ priors,}
  \AttributeTok{chains  =} \DecValTok{4}\NormalTok{, }\AttributeTok{iter =} \DecValTok{2000}\NormalTok{, }\AttributeTok{seed =} \DecValTok{123}\NormalTok{,}
  \AttributeTok{control =} \FunctionTok{list}\NormalTok{(}\AttributeTok{adapt\_delta =} \FloatTok{0.95}\NormalTok{),}
  \AttributeTok{refresh =} \DecValTok{0}
\NormalTok{)}
\end{Highlighting}
\end{Shaded}

\begin{verbatim}
Running MCMC with 4 sequential chains...

Chain 1 finished in 11.3 seconds.
Chain 2 finished in 10.4 seconds.
Chain 3 finished in 10.8 seconds.
Chain 4 finished in 11.4 seconds.

All 4 chains finished successfully.
Mean chain execution time: 11.0 seconds.
Total execution time: 44.4 seconds.
\end{verbatim}

\begin{Shaded}
\begin{Highlighting}[]
\FunctionTok{summary}\NormalTok{(bayes\_fit)}
\end{Highlighting}
\end{Shaded}

\begin{verbatim}
 Family: bernoulli 
  Links: mu = logit 
Formula: diabetes_dx | weights(wt_norm) ~ age_c + bmi_c + sex + race 
   Data: adult_imp1 (Number of observations: 5592) 
  Draws: 4 chains, each with iter = 2000; warmup = 1000; thin = 1;
         total post-warmup draws = 4000

Regression Coefficients:
                    Estimate Est.Error l-95% CI u-95% CI Rhat Bulk_ESS Tail_ESS
Intercept              -2.66      0.09    -2.83    -2.50 1.00     3548     3512
age_c                   1.10      0.06     0.98     1.22 1.00     2349     2618
bmi_c                   0.63      0.05     0.54     0.72 1.00     3327     2826
sexFemale              -0.66      0.10    -0.86    -0.47 1.00     3668     3124
raceMexicanAmerican     0.69      0.17     0.34     1.03 1.00     3657     2821
raceOtherHispanic       0.43      0.25    -0.07     0.89 1.00     4242     3014
raceNHBlack             0.53      0.15     0.23     0.83 1.00     3809     3012
raceOtherDMulti         0.81      0.19     0.45     1.18 1.00     3948     2809

Draws were sampled using sample(hmc). For each parameter, Bulk_ESS
and Tail_ESS are effective sample size measures, and Rhat is the potential
scale reduction factor on split chains (at convergence, Rhat = 1).
\end{verbatim}

Bayesian logistic regression model fit summary for diabetes diagnosis
(\texttt{diabetes\_dx}) with standardized predictors (age, BMI, sex, and
race) and normalized NHANES weights. All four MCMC chains (4,000
post-warmup draws) converged successfully (\texttt{R̂\ ≈\ 1.00}),
indicating stable estimation across parameters.

\begin{Shaded}
\begin{Highlighting}[]
\CommentTok{\# Extract fixed effects and convert to odds ratios}
\NormalTok{bayes\_fixef }\OtherTok{\textless{}{-}}\NormalTok{ brms}\SpecialCharTok{::}\FunctionTok{fixef}\NormalTok{(bayes\_fit, }\AttributeTok{summary =} \ConstantTok{TRUE}\NormalTok{)}

\NormalTok{bayes\_or }\OtherTok{\textless{}{-}}\NormalTok{ bayes\_fixef }\SpecialCharTok{\%\textgreater{}\%}
  \FunctionTok{as.data.frame}\NormalTok{() }\SpecialCharTok{\%\textgreater{}\%}
\NormalTok{  tibble}\SpecialCharTok{::}\FunctionTok{rownames\_to\_column}\NormalTok{(}\StringTok{"term"}\NormalTok{) }\SpecialCharTok{\%\textgreater{}\%}
\NormalTok{  dplyr}\SpecialCharTok{::}\FunctionTok{mutate}\NormalTok{(}
    \AttributeTok{OR  =} \FunctionTok{exp}\NormalTok{(Estimate),}
    \AttributeTok{LCL =} \FunctionTok{exp}\NormalTok{(Q2}\FloatTok{.5}\NormalTok{),}
    \AttributeTok{UCL =} \FunctionTok{exp}\NormalTok{(Q97}\FloatTok{.5}\NormalTok{)}
\NormalTok{  )}
\end{Highlighting}
\end{Shaded}

\paragraph{Posterior Odd Ratios (Main
Results)}\label{posterior-odd-ratios-main-results}

\begin{Shaded}
\begin{Highlighting}[]
\NormalTok{knitr}\SpecialCharTok{::}\FunctionTok{kable}\NormalTok{(}
\NormalTok{dplyr}\SpecialCharTok{::}\FunctionTok{mutate}\NormalTok{(bayes\_or, dplyr}\SpecialCharTok{::}\FunctionTok{across}\NormalTok{(}\FunctionTok{c}\NormalTok{(OR, LCL, UCL), }\SpecialCharTok{\textasciitilde{}} \FunctionTok{round}\NormalTok{(.x, }\DecValTok{2}\NormalTok{)))}
\NormalTok{)}
\end{Highlighting}
\end{Shaded}

\begin{longtable}[]{@{}
  >{\raggedright\arraybackslash}p{(\linewidth - 14\tabcolsep) * \real{0.2564}}
  >{\raggedleft\arraybackslash}p{(\linewidth - 14\tabcolsep) * \real{0.1410}}
  >{\raggedleft\arraybackslash}p{(\linewidth - 14\tabcolsep) * \real{0.1282}}
  >{\raggedleft\arraybackslash}p{(\linewidth - 14\tabcolsep) * \real{0.1410}}
  >{\raggedleft\arraybackslash}p{(\linewidth - 14\tabcolsep) * \real{0.1410}}
  >{\raggedleft\arraybackslash}p{(\linewidth - 14\tabcolsep) * \real{0.0641}}
  >{\raggedleft\arraybackslash}p{(\linewidth - 14\tabcolsep) * \real{0.0641}}
  >{\raggedleft\arraybackslash}p{(\linewidth - 14\tabcolsep) * \real{0.0641}}@{}}

\caption{\label{tbl-bayes}}

\tabularnewline

\toprule\noalign{}
\begin{minipage}[b]{\linewidth}\raggedright
term
\end{minipage} & \begin{minipage}[b]{\linewidth}\raggedleft
Estimate
\end{minipage} & \begin{minipage}[b]{\linewidth}\raggedleft
Est.Error
\end{minipage} & \begin{minipage}[b]{\linewidth}\raggedleft
Q2.5
\end{minipage} & \begin{minipage}[b]{\linewidth}\raggedleft
Q97.5
\end{minipage} & \begin{minipage}[b]{\linewidth}\raggedleft
OR
\end{minipage} & \begin{minipage}[b]{\linewidth}\raggedleft
LCL
\end{minipage} & \begin{minipage}[b]{\linewidth}\raggedleft
UCL
\end{minipage} \\
\midrule\noalign{}
\endhead
\bottomrule\noalign{}
\endlastfoot
Intercept & -2.6633187 & 0.0868613 & -2.8341138 & -2.4958967 & 0.07 &
0.06 & 0.08 \\
age\_c & 1.0968784 & 0.0618886 & 0.9783744 & 1.2200119 & 2.99 & 2.66 &
3.39 \\
bmi\_c & 0.6282273 & 0.0467939 & 0.5366821 & 0.7199012 & 1.87 & 1.71 &
2.05 \\
sexFemale & -0.6624742 & 0.1034594 & -0.8645869 & -0.4660003 & 0.52 &
0.42 & 0.63 \\
raceMexicanAmerican & 0.6898163 & 0.1710160 & 0.3432716 & 1.0298163 &
1.99 & 1.41 & 2.80 \\
raceOtherHispanic & 0.4252184 & 0.2458586 & -0.0669575 & 0.8870126 &
1.53 & 0.94 & 2.43 \\
raceNHBlack & 0.5307334 & 0.1524774 & 0.2283617 & 0.8328511 & 1.70 &
1.26 & 2.30 \\
raceOtherDMulti & 0.8143883 & 0.1876762 & 0.4467512 & 1.1763335 & 2.26 &
1.56 & 3.24 \\

\end{longtable}

\begin{itemize}
\item
  Age and BMI show strong positive associations with diabetes (credible
  intervals exclude 1).
\item
  Female sex shows lower odds than male (protective factor).
\item
  Non-White racial groups have higher odds compared with Whites,
  consistent with known disparities.
\item
  All model parameters exhibit well-defined, unimodal posteriors with
  narrow credible intervals.
\end{itemize}

\paragraph{Diagnostics and Model Fit}\label{diagnostics-and-model-fit}

\begin{Shaded}
\begin{Highlighting}[]
\NormalTok{knitr}\SpecialCharTok{::}\FunctionTok{kable}\NormalTok{(}\FunctionTok{as.data.frame}\NormalTok{(brms}\SpecialCharTok{::}\FunctionTok{bayes\_R2}\NormalTok{(bayes\_fit)))}
\end{Highlighting}
\end{Shaded}

\begin{longtable}[]{@{}lrrrr@{}}

\caption{\label{tbl-bayesR2}Bayesian R² Summary}

\tabularnewline

\toprule\noalign{}
& Estimate & Est.Error & Q2.5 & Q97.5 \\
\midrule\noalign{}
\endhead
\bottomrule\noalign{}
\endlastfoot
R2 & 0.1316278 & 0.0123417 & 0.107432 & 0.1565549 \\

\end{longtable}

\begin{Shaded}
\begin{Highlighting}[]
\NormalTok{diag }\OtherTok{\textless{}{-}}\NormalTok{ posterior}\SpecialCharTok{::}\FunctionTok{summarise\_draws}\NormalTok{(bayes\_fit, }\StringTok{"rhat"}\NormalTok{, }\StringTok{"ess\_bulk"}\NormalTok{, }\StringTok{"ess\_tail"}\NormalTok{)}

\NormalTok{diag\_b }\OtherTok{\textless{}{-}}\NormalTok{ diag }\SpecialCharTok{|\textgreater{}}
\NormalTok{dplyr}\SpecialCharTok{::}\FunctionTok{as\_tibble}\NormalTok{() }\SpecialCharTok{|\textgreater{}}
\NormalTok{dplyr}\SpecialCharTok{::}\FunctionTok{filter}\NormalTok{(}\FunctionTok{grepl}\NormalTok{(}\StringTok{"\^{}b\_"}\NormalTok{, .data}\SpecialCharTok{$}\NormalTok{variable)) }\SpecialCharTok{|\textgreater{}}
\NormalTok{dplyr}\SpecialCharTok{::}\FunctionTok{transmute}\NormalTok{(}
\AttributeTok{Parameter =}\NormalTok{ .data}\SpecialCharTok{$}\NormalTok{variable,}
\AttributeTok{Rhat      =}\NormalTok{ .data}\SpecialCharTok{$}\NormalTok{rhat,}
\AttributeTok{Bulk\_ESS  =}\NormalTok{ .data}\SpecialCharTok{$}\NormalTok{ess\_bulk,}
\AttributeTok{Tail\_ESS  =}\NormalTok{ .data}\SpecialCharTok{$}\NormalTok{ess\_tail}
\NormalTok{)}

\NormalTok{knitr}\SpecialCharTok{::}\FunctionTok{kable}\NormalTok{(diag\_b, }\AttributeTok{digits =} \DecValTok{1}\NormalTok{)}
\end{Highlighting}
\end{Shaded}

\begin{longtable}[]{@{}lrrr@{}}

\caption{\label{tbl-mcmc-diagnostics}MCMC Diagnostics (R-hat and
Effective Sample Sizes) for Model Parameters}

\tabularnewline

\toprule\noalign{}
Parameter & Rhat & Bulk\_ESS & Tail\_ESS \\
\midrule\noalign{}
\endhead
\bottomrule\noalign{}
\endlastfoot
b\_Intercept & 1 & 3548.0 & 3511.8 \\
b\_age\_c & 1 & 2349.3 & 2617.8 \\
b\_bmi\_c & 1 & 3327.1 & 2825.9 \\
b\_sexFemale & 1 & 3668.1 & 3123.7 \\
b\_raceMexicanAmerican & 1 & 3656.6 & 2821.2 \\
b\_raceOtherHispanic & 1 & 4242.3 & 3013.5 \\
b\_raceNHBlack & 1 & 3809.1 & 3012.2 \\
b\_raceOtherDMulti & 1 & 3947.9 & 2809.1 \\

\end{longtable}

All parameters achieved R̂ ≈ 1.00 and effective sample sizes
\textgreater2,000, indicating excellent convergence. The Bayesian R² ≈
0.13, showing that age, BMI, sex, and race explain about 13\% of
diabetes variability.

\paragraph{Model Comparison}\label{model-comparison}

\begin{Shaded}
\begin{Highlighting}[]
\FunctionTok{invisible}\NormalTok{(}\FunctionTok{capture.output}\NormalTok{(\{}
\NormalTok{fit\_no\_race }\OtherTok{\textless{}{-}} \FunctionTok{update}\NormalTok{(bayes\_fit, }\AttributeTok{formula =} \FunctionTok{update}\NormalTok{(fml\_bayes, . }\SpecialCharTok{\textasciitilde{}}\NormalTok{ . }\SpecialCharTok{{-}}\NormalTok{ race))}
\NormalTok{fit\_no\_sex  }\OtherTok{\textless{}{-}} \FunctionTok{update}\NormalTok{(bayes\_fit, }\AttributeTok{formula =} \FunctionTok{update}\NormalTok{(fml\_bayes, . }\SpecialCharTok{\textasciitilde{}}\NormalTok{ . }\SpecialCharTok{{-}}\NormalTok{ sex))}
\NormalTok{\}))}

\NormalTok{loo\_base    }\OtherTok{\textless{}{-}}\NormalTok{ loo}\SpecialCharTok{::}\FunctionTok{loo}\NormalTok{(bayes\_fit)}
\NormalTok{loo\_no\_race }\OtherTok{\textless{}{-}}\NormalTok{ loo}\SpecialCharTok{::}\FunctionTok{loo}\NormalTok{(fit\_no\_race)}
\NormalTok{loo\_no\_sex  }\OtherTok{\textless{}{-}}\NormalTok{ loo}\SpecialCharTok{::}\FunctionTok{loo}\NormalTok{(fit\_no\_sex)}

\NormalTok{cmp\_df }\OtherTok{\textless{}{-}} \FunctionTok{as.data.frame}\NormalTok{(loo}\SpecialCharTok{::}\FunctionTok{loo\_compare}\NormalTok{(loo\_base, loo\_no\_race, loo\_no\_sex))}
\NormalTok{cmp\_df}\SpecialCharTok{$}\NormalTok{Model }\OtherTok{\textless{}{-}} \FunctionTok{rownames}\NormalTok{(cmp\_df)}
\NormalTok{cmp\_df }\OtherTok{\textless{}{-}}\NormalTok{ cmp\_df[, }\FunctionTok{c}\NormalTok{(}\StringTok{"Model"}\NormalTok{, }\FunctionTok{setdiff}\NormalTok{(}\FunctionTok{names}\NormalTok{(cmp\_df), }\StringTok{"Model"}\NormalTok{))]}

\NormalTok{knitr}\SpecialCharTok{::}\FunctionTok{kable}\NormalTok{(}
\NormalTok{cmp\_df,}
\AttributeTok{caption =} \StringTok{"LOO Comparison (higher elpd\_loo indicates better predictive performance)."}
\NormalTok{)}
\end{Highlighting}
\end{Shaded}

\begin{longtable}[]{@{}
  >{\raggedright\arraybackslash}p{(\linewidth - 18\tabcolsep) * \real{0.1176}}
  >{\raggedright\arraybackslash}p{(\linewidth - 18\tabcolsep) * \real{0.1176}}
  >{\raggedleft\arraybackslash}p{(\linewidth - 18\tabcolsep) * \real{0.0980}}
  >{\raggedleft\arraybackslash}p{(\linewidth - 18\tabcolsep) * \real{0.0882}}
  >{\raggedleft\arraybackslash}p{(\linewidth - 18\tabcolsep) * \real{0.0980}}
  >{\raggedleft\arraybackslash}p{(\linewidth - 18\tabcolsep) * \real{0.1176}}
  >{\raggedleft\arraybackslash}p{(\linewidth - 18\tabcolsep) * \real{0.0882}}
  >{\raggedleft\arraybackslash}p{(\linewidth - 18\tabcolsep) * \real{0.0980}}
  >{\raggedleft\arraybackslash}p{(\linewidth - 18\tabcolsep) * \real{0.0882}}
  >{\raggedleft\arraybackslash}p{(\linewidth - 18\tabcolsep) * \real{0.0882}}@{}}
\caption{Bayesian Model Comparison (LOO): Base Model vs.~Reduced Models
Without Race or Sex}\tabularnewline
\toprule\noalign{}
\begin{minipage}[b]{\linewidth}\raggedright
\end{minipage} & \begin{minipage}[b]{\linewidth}\raggedright
Model
\end{minipage} & \begin{minipage}[b]{\linewidth}\raggedleft
elpd\_diff
\end{minipage} & \begin{minipage}[b]{\linewidth}\raggedleft
se\_diff
\end{minipage} & \begin{minipage}[b]{\linewidth}\raggedleft
elpd\_loo
\end{minipage} & \begin{minipage}[b]{\linewidth}\raggedleft
se\_elpd\_loo
\end{minipage} & \begin{minipage}[b]{\linewidth}\raggedleft
p\_loo
\end{minipage} & \begin{minipage}[b]{\linewidth}\raggedleft
se\_p\_loo
\end{minipage} & \begin{minipage}[b]{\linewidth}\raggedleft
looic
\end{minipage} & \begin{minipage}[b]{\linewidth}\raggedleft
se\_looic
\end{minipage} \\
\midrule\noalign{}
\endfirsthead
\toprule\noalign{}
\begin{minipage}[b]{\linewidth}\raggedright
\end{minipage} & \begin{minipage}[b]{\linewidth}\raggedright
Model
\end{minipage} & \begin{minipage}[b]{\linewidth}\raggedleft
elpd\_diff
\end{minipage} & \begin{minipage}[b]{\linewidth}\raggedleft
se\_diff
\end{minipage} & \begin{minipage}[b]{\linewidth}\raggedleft
elpd\_loo
\end{minipage} & \begin{minipage}[b]{\linewidth}\raggedleft
se\_elpd\_loo
\end{minipage} & \begin{minipage}[b]{\linewidth}\raggedleft
p\_loo
\end{minipage} & \begin{minipage}[b]{\linewidth}\raggedleft
se\_p\_loo
\end{minipage} & \begin{minipage}[b]{\linewidth}\raggedleft
looic
\end{minipage} & \begin{minipage}[b]{\linewidth}\raggedleft
se\_looic
\end{minipage} \\
\midrule\noalign{}
\endhead
\bottomrule\noalign{}
\endlastfoot
bayes\_fit & bayes\_fit & 0.00000 & 0.000000 & -1418.258 & 56.42097 &
8.732434 & 0.5944729 & 2836.517 & 112.8419 \\
fit\_no\_race & fit\_no\_race & -14.43171 & 6.367627 & -1432.690 &
53.98749 & 5.223838 & 0.3831466 & 2865.380 & 107.9750 \\
fit\_no\_sex & fit\_no\_sex & -20.04611 & 8.205833 & -1438.305 &
57.31024 & 7.359525 & 0.5226182 & 2876.609 & 114.6205 \\
\end{longtable}

Models excluding race or sex had lower expected log predictive density
(\texttt{elpd}), confirming that both variables contribute meaningfully
to model fit.

\paragraph{Posterior Predictive
Checks}\label{posterior-predictive-checks}

\begin{Shaded}
\begin{Highlighting}[]
\NormalTok{yobs }\OtherTok{\textless{}{-}}\NormalTok{ adult\_imp1}\SpecialCharTok{$}\NormalTok{diabetes\_dx}
\end{Highlighting}
\end{Shaded}

\begin{Shaded}
\begin{Highlighting}[]
\NormalTok{bayesplot}\SpecialCharTok{::}\FunctionTok{pp\_check}\NormalTok{(bayes\_fit, }\AttributeTok{type =} \StringTok{"bars"}\NormalTok{, }\AttributeTok{nsamples =} \DecValTok{100}\NormalTok{)}
\end{Highlighting}
\end{Shaded}

\begin{figure}[H]

\centering{

\pandocbounded{\includegraphics[keepaspectratio]{index_files/figure-pdf/fig-ppc-bars-1.pdf}}

}

\caption{\label{fig-ppc-bars}Posterior Predictive Check: Observed
vs.~Replicated Outcome Distribution (Bars)}

\end{figure}%

The close alignment between observed (\texttt{y}) and replicated
(\texttt{y\_rep}) outcome distributions indicates that the Bayesian
model reproduces the empirical data structure well.

\begin{Shaded}
\begin{Highlighting}[]
\NormalTok{yrep }\OtherTok{\textless{}{-}}\NormalTok{ brms}\SpecialCharTok{::}\FunctionTok{posterior\_predict}\NormalTok{(bayes\_fit, }\AttributeTok{ndraws =} \DecValTok{400}\NormalTok{)}
\NormalTok{bayesplot}\SpecialCharTok{::}\FunctionTok{ppc\_stat}\NormalTok{(}\AttributeTok{y =}\NormalTok{ yobs, }\AttributeTok{yrep =}\NormalTok{ yrep, }\AttributeTok{stat =} \StringTok{"mean"}\NormalTok{)}
\end{Highlighting}
\end{Shaded}

\begin{figure}[H]

\centering{

\pandocbounded{\includegraphics[keepaspectratio]{index_files/figure-pdf/fig-ppc-mean-1.pdf}}

}

\caption{\label{fig-ppc-mean}Posterior predictive check for the mean of
the binary outcome, comparing the observed mean (\texttt{T(y)}) to
replicated means (\texttt{T(y\_rep)}) across posterior draws.}

\end{figure}%

\begin{Shaded}
\begin{Highlighting}[]
\NormalTok{yrep }\OtherTok{\textless{}{-}}\NormalTok{ brms}\SpecialCharTok{::}\FunctionTok{posterior\_predict}\NormalTok{(bayes\_fit, }\AttributeTok{ndraws =} \DecValTok{400}\NormalTok{)}
\NormalTok{bayesplot}\SpecialCharTok{::}\FunctionTok{ppc\_stat}\NormalTok{(}\AttributeTok{y =}\NormalTok{ yobs, }\AttributeTok{yrep =}\NormalTok{ yrep, }\AttributeTok{stat =} \StringTok{"sd"}\NormalTok{)}
\end{Highlighting}
\end{Shaded}

\begin{figure}[H]

\centering{

\pandocbounded{\includegraphics[keepaspectratio]{index_files/figure-pdf/fig-ppc-sd-1.pdf}}

}

\caption{\label{fig-ppc-sd}Posterior predictive check for the standard
deviation of the binary outcome (\texttt{T(y)}) compared with replicated
datasets (\texttt{T(y\_rep)}).}

\end{figure}%

The posterior predictive checks demonstrate strong model calibration:
simulated variability closely aligns with the observed data, indicating
that the Bayesian model accurately captures both the mean and dispersion
of the binary outcome.

\paragraph{MCMC Diagnostics and Posterior
Distributions}\label{mcmc-diagnostics-and-posterior-distributions}

\begin{Shaded}
\begin{Highlighting}[]
\NormalTok{bayesplot}\SpecialCharTok{::}\FunctionTok{mcmc\_areas}\NormalTok{(}\FunctionTok{as.array}\NormalTok{(bayes\_fit), }\AttributeTok{regex\_pars =} \StringTok{"\^{}b\_"}\NormalTok{, }\AttributeTok{prob =} \FloatTok{0.95}\NormalTok{)}
\end{Highlighting}
\end{Shaded}

\begin{figure}[H]

\centering{

\pandocbounded{\includegraphics[keepaspectratio]{index_files/figure-pdf/fig-mcmc-areas-1.pdf}}

}

\caption{\label{fig-mcmc-areas}Posterior distributions (95\% credible
mass) for slope parameters in the Bayesian logistic regression model.}

\end{figure}%

All posteriors appear unimodal and well‐centered, indicating stable
estimation and strong convergence across parameters. Positive
coefficients (e.g., age, BMI) correspond to increased diabetes risk,
while negative coefficients (e.g., female sex) indicate protective
associations.

\begin{Shaded}
\begin{Highlighting}[]
\NormalTok{bayesplot}\SpecialCharTok{::}\FunctionTok{mcmc\_trace}\NormalTok{(}\FunctionTok{as.array}\NormalTok{(bayes\_fit), }\AttributeTok{regex\_pars =} \StringTok{"\^{}b\_"}\NormalTok{)}
\end{Highlighting}
\end{Shaded}

\begin{figure}[H]

\centering{

\pandocbounded{\includegraphics[keepaspectratio]{index_files/figure-pdf/fig-mcmc-trace-1.pdf}}

}

\caption{\label{fig-mcmc-trace}Trace plots for slope parameters across
four MCMC chains, demonstrating effective chain mixing and
stationarity.}

\end{figure}%

All parameters exhibit well-mixed, stable trace patterns with no visible
drift, supporting convergence diagnostics (\texttt{R̂} ≈ 1.00). This
confirms that the posterior samples are representative and that the
Bayesian model converged reliably.

\begin{Shaded}
\begin{Highlighting}[]
\NormalTok{post\_array }\OtherTok{\textless{}{-}}\NormalTok{ posterior}\SpecialCharTok{::}\FunctionTok{as\_draws\_array}\NormalTok{(bayes\_fit)}
\NormalTok{bayesplot}\SpecialCharTok{::}\FunctionTok{mcmc\_acf}\NormalTok{(post\_array, }\AttributeTok{pars =} \FunctionTok{c}\NormalTok{(}\StringTok{"b\_age\_c"}\NormalTok{, }\StringTok{"b\_bmi\_c"}\NormalTok{))}
\end{Highlighting}
\end{Shaded}

\begin{figure}[H]

\centering{

\pandocbounded{\includegraphics[keepaspectratio]{index_files/figure-pdf/fig-mcmc-acf-1.pdf}}

}

\caption{\label{fig-mcmc-acf}Autocorrelation plots for posterior samples
of age and BMI coefficients, showing rapid decay of autocorrelation with
lag. Low autocorrelation across lags confirms efficient MCMC sampling
and good chain independence.}

\end{figure}%

\begin{itemize}
\item
  Trace, density, and autocorrelation plots confirm smooth chain mixing,
  unimodal posteriors, and minimal autocorrelation across samples.
\item
  All four chains showed strong convergence with no signs of divergence
  or non-stationarity.
\item
  Trace plots revealed stable, overlapping chains with consistent mixing
  across iterations, while autocorrelation decayed rapidly toward zero,
  confirming efficient sampling and low dependency between successive
  draws.
\item
  Together with R̂ ≈ 1.00 and large effective sample sizes, these
  diagnostics indicate a well-behaved posterior and reliable inference.
\end{itemize}

\paragraph{Prior vs.~Posterior}\label{prior-vs.-posterior}

\begin{Shaded}
\begin{Highlighting}[]
\CommentTok{\# Extract posterior draws as a matrix, then convert to tibble}
\NormalTok{post }\OtherTok{\textless{}{-}} \FunctionTok{as\_draws\_matrix}\NormalTok{(bayes\_fit) }\SpecialCharTok{\%\textgreater{}\%}   \CommentTok{\# safer than as\_draws\_df for manipulation}
  \FunctionTok{as.data.frame}\NormalTok{() }\SpecialCharTok{\%\textgreater{}\%}
  \FunctionTok{select}\NormalTok{(b\_bmi\_c, b\_age\_c) }\SpecialCharTok{\%\textgreater{}\%}
  \FunctionTok{pivot\_longer}\NormalTok{(}
    \FunctionTok{everything}\NormalTok{(),}
    \AttributeTok{names\_to =} \StringTok{"term"}\NormalTok{,}
    \AttributeTok{values\_to =} \StringTok{"estimate"}
\NormalTok{  ) }\SpecialCharTok{\%\textgreater{}\%}
  \FunctionTok{mutate}\NormalTok{(}
    \AttributeTok{term =} \FunctionTok{case\_when}\NormalTok{(}
\NormalTok{      term }\SpecialCharTok{==} \StringTok{"b\_bmi\_c"} \SpecialCharTok{\textasciitilde{}} \StringTok{"BMI (per 1 SD)"}\NormalTok{,}
\NormalTok{      term }\SpecialCharTok{==} \StringTok{"b\_age\_c"} \SpecialCharTok{\textasciitilde{}} \StringTok{"Age (per 1 SD)"}
\NormalTok{    ),}
    \AttributeTok{type =} \StringTok{"Posterior"}
\NormalTok{  )}
\NormalTok{prior\_draws }\OtherTok{\textless{}{-}} \FunctionTok{tibble}\NormalTok{(}
  \AttributeTok{term =} \FunctionTok{rep}\NormalTok{(}\FunctionTok{c}\NormalTok{(}\StringTok{"BMI (per 1 SD)"}\NormalTok{, }\StringTok{"Age (per 1 SD)"}\NormalTok{), }\AttributeTok{each =} \DecValTok{4000}\NormalTok{),}
  \AttributeTok{estimate =} \FunctionTok{c}\NormalTok{(}\FunctionTok{rnorm}\NormalTok{(}\DecValTok{4000}\NormalTok{, }\DecValTok{0}\NormalTok{, }\DecValTok{1}\NormalTok{), }\FunctionTok{rnorm}\NormalTok{(}\DecValTok{4000}\NormalTok{, }\DecValTok{0}\NormalTok{, }\DecValTok{1}\NormalTok{)),}
  \AttributeTok{type =} \StringTok{"Prior"}
\NormalTok{)}
\NormalTok{combined\_draws }\OtherTok{\textless{}{-}} \FunctionTok{bind\_rows}\NormalTok{(prior\_draws, post)}

\FunctionTok{ggplot}\NormalTok{(combined\_draws, }\FunctionTok{aes}\NormalTok{(}\AttributeTok{x =}\NormalTok{ estimate, }\AttributeTok{fill =}\NormalTok{ type)) }\SpecialCharTok{+}
  \FunctionTok{geom\_density}\NormalTok{(}\AttributeTok{alpha =} \FloatTok{0.4}\NormalTok{) }\SpecialCharTok{+}
  \FunctionTok{facet\_wrap}\NormalTok{(}\SpecialCharTok{\textasciitilde{}}\NormalTok{ term, }\AttributeTok{scales =} \StringTok{"free"}\NormalTok{, }\AttributeTok{ncol =} \DecValTok{2}\NormalTok{) }\SpecialCharTok{+}
  \FunctionTok{theme\_minimal}\NormalTok{(}\AttributeTok{base\_size =} \DecValTok{13}\NormalTok{) }\SpecialCharTok{+}
  \FunctionTok{labs}\NormalTok{(}
    \AttributeTok{title =} \StringTok{"Prior vs Posterior Distributions"}\NormalTok{,}
    \AttributeTok{x =} \StringTok{"Coefficient estimate"}\NormalTok{,}
    \AttributeTok{y =} \StringTok{"Density"}\NormalTok{,}
    \AttributeTok{fill =} \StringTok{""}
\NormalTok{  )}
\end{Highlighting}
\end{Shaded}

\begin{figure}[H]

\centering{

\pandocbounded{\includegraphics[keepaspectratio]{index_files/figure-pdf/fig-prior-posterior-ggplot-2-1.pdf}}

}

\caption{\label{fig-prior-posterior-ggplot-2}Prior vs Posterior
Distributions}

\end{figure}%

For age and BMI, the posterior densities shift notably away from the
N(0, 2.5) prior toward positive values and are narrower, indicating
strong information from the data; for sex, the posterior remains closer
to the prior with more overlap, indicating weaker evidence.

The overlay of prior and posterior densities illustrates that
informative updates occurred primarily for BMI, age, and race
coefficients, which showed distinct posterior shifts relative to the
priors. In contrast, weaker predictors such as sex displayed overlapping
distributions, indicating that inference for those parameters was more
influenced by prior uncertainty than by the observed data. This balance
confirms appropriate regularization rather than overfitting.

\paragraph{Model Fit and Calibration}\label{model-fit-and-calibration}

\begin{Shaded}
\begin{Highlighting}[]
\NormalTok{pred\_mean }\OtherTok{\textless{}{-}} \FunctionTok{colMeans}\NormalTok{(brms}\SpecialCharTok{::}\FunctionTok{posterior\_epred}\NormalTok{(bayes\_fit))}
\FunctionTok{ggplot}\NormalTok{(}\FunctionTok{data.frame}\NormalTok{(}\AttributeTok{pred =}\NormalTok{ pred\_mean, }\AttributeTok{obs =}\NormalTok{ yobs),}
\FunctionTok{aes}\NormalTok{(}\AttributeTok{x =}\NormalTok{ pred, }\AttributeTok{y =}\NormalTok{ obs)) }\SpecialCharTok{+}
\FunctionTok{geom\_point}\NormalTok{(}\AttributeTok{alpha =} \FloatTok{0.15}\NormalTok{, }\AttributeTok{position =} \FunctionTok{position\_jitter}\NormalTok{(}\AttributeTok{height =} \FloatTok{0.03}\NormalTok{)) }\SpecialCharTok{+}
\FunctionTok{geom\_smooth}\NormalTok{(}\AttributeTok{method =} \StringTok{"loess"}\NormalTok{, }\AttributeTok{se =} \ConstantTok{TRUE}\NormalTok{) }\SpecialCharTok{+}
\FunctionTok{labs}\NormalTok{(}\AttributeTok{x =} \StringTok{"Mean predicted probability"}\NormalTok{, }\AttributeTok{y =} \StringTok{"Observed diabetes (0/1)"}\NormalTok{)}
\end{Highlighting}
\end{Shaded}

\begin{figure}[H]

\centering{

\pandocbounded{\includegraphics[keepaspectratio]{index_files/figure-pdf/fig-pred-calibration-1.pdf}}

}

\caption{\label{fig-pred-calibration}Calibration plot comparing observed
diabetes outcomes (0/1) to model-predicted probabilities with a smoothed
LOESS curve. The close alignment between the blue line and the diagonal
(ideal calibration) indicates good model fit and reliable probability
estimates.}

\end{figure}%

\begin{Shaded}
\begin{Highlighting}[]
\CommentTok{\# 1. Survey{-}weighted prevalence}
\NormalTok{svy\_mean }\OtherTok{\textless{}{-}} \FunctionTok{svymean}\NormalTok{(}\SpecialCharTok{\textasciitilde{}}\NormalTok{diabetes\_dx, nhanes\_design\_adult, }\AttributeTok{na.rm =} \ConstantTok{TRUE}\NormalTok{)}

\CommentTok{\# 2. Posterior predictive prevalence (per draw)}
\NormalTok{pp\_samples }\OtherTok{\textless{}{-}}\NormalTok{ brms}\SpecialCharTok{::}\FunctionTok{posterior\_predict}\NormalTok{(bayes\_fit, }\AttributeTok{ndraws =} \DecValTok{1000}\NormalTok{)  }\CommentTok{\# draws x individuals}
\NormalTok{pp\_proportion }\OtherTok{\textless{}{-}} \FunctionTok{rowMeans}\NormalTok{(pp\_samples)                            }\CommentTok{\# prevalence per draw}

\CommentTok{\# 3. Build comparison table}
\NormalTok{summary\_table }\OtherTok{\textless{}{-}} \FunctionTok{tibble}\NormalTok{(}
  \AttributeTok{Method =} \FunctionTok{c}\NormalTok{(}\StringTok{"Survey{-}weighted mean (NHANES)"}\NormalTok{, }
             \StringTok{"Imputed dataset mean"}\NormalTok{, }
             \StringTok{"Posterior predictive mean"}\NormalTok{),}
  \AttributeTok{diabetes\_mean =} \FunctionTok{c}\NormalTok{(}
    \FunctionTok{coef}\NormalTok{(svy\_mean),                           }\CommentTok{\# survey{-}weighted mean}
    \FunctionTok{mean}\NormalTok{(adult\_imp1}\SpecialCharTok{$}\NormalTok{diabetes\_dx, }\AttributeTok{na.rm =} \ConstantTok{TRUE}\NormalTok{),  }\CommentTok{\# imputed dataset}
    \FunctionTok{mean}\NormalTok{(pp\_proportion)                       }\CommentTok{\# posterior predictive mean}
\NormalTok{  ),}
  \AttributeTok{SE =} \FunctionTok{c}\NormalTok{(}
    \FunctionTok{SE}\NormalTok{(svy\_mean),   }\CommentTok{\# survey{-}weighted SE}
    \ConstantTok{NA}\NormalTok{,             }\CommentTok{\# not available for raw mean}
    \ConstantTok{NA}              \CommentTok{\# not available for posterior predictive mean}
\NormalTok{  )}
\NormalTok{)}

\FunctionTok{kable}\NormalTok{(summary\_table, }\AttributeTok{digits =} \DecValTok{4}\NormalTok{,}
      \AttributeTok{caption =} \StringTok{"Comparison of Diabetes Prevalence Across Methods"}\NormalTok{)}
\end{Highlighting}
\end{Shaded}

\begin{figure}

\centering{

\begin{longtable}[]{@{}lrr@{}}
\caption{Comparison of Diabetes Prevalence Across
Methods}\tabularnewline
\toprule\noalign{}
Method & diabetes\_mean & SE \\
\midrule\noalign{}
\endfirsthead
\toprule\noalign{}
Method & diabetes\_mean & SE \\
\midrule\noalign{}
\endhead
\bottomrule\noalign{}
\endlastfoot
Survey-weighted mean (NHANES) & 0.0889 & 0.0048 \\
Imputed dataset mean & 0.1105 & NA \\
Posterior predictive mean & 0.1093 & NA \\
\end{longtable}

}

\caption{\label{fig-posterior-prevalence}Comparison of diabetes
prevalence across survey-weighted (NHANES), imputed, and posterior
predictive estimates. The posterior predictive mean aligns closely with
the observed NHANES prevalence, indicating strong model calibration.}

\end{figure}%

The posterior predictive distribution of diabetes prevalence closely
mirrored the survey-estimated prevalence, with the posterior mean
aligning within about 1\% of the observed rate.

\begin{Shaded}
\begin{Highlighting}[]
\CommentTok{\# Posterior predictive prevalence (replicated datasets)}

\NormalTok{yrep }\OtherTok{\textless{}{-}}\NormalTok{ brms}\SpecialCharTok{::}\FunctionTok{posterior\_predict}\NormalTok{(bayes\_fit, }\AttributeTok{ndraws =} \DecValTok{2000}\NormalTok{)   }\CommentTok{\# draws x observations (0/1)}
\NormalTok{post\_prev }\OtherTok{\textless{}{-}} \FunctionTok{rowMeans}\NormalTok{(yrep)                                 }\CommentTok{\# prevalence each posterior draw}

\CommentTok{\# Survey{-}weighted observed prevalence (population estimate)}

\NormalTok{des\_obs }\OtherTok{\textless{}{-}}\NormalTok{ survey}\SpecialCharTok{::}\FunctionTok{svydesign}\NormalTok{(}
\AttributeTok{id =} \SpecialCharTok{\textasciitilde{}}\NormalTok{SDMVPSU, }\AttributeTok{strata =} \SpecialCharTok{\textasciitilde{}}\NormalTok{SDMVSTRA, }\AttributeTok{weights =} \SpecialCharTok{\textasciitilde{}}\NormalTok{WTMEC2YR,}
\AttributeTok{nest =} \ConstantTok{TRUE}\NormalTok{, }\AttributeTok{data =}\NormalTok{ adult\_imp1}
\NormalTok{)}
\NormalTok{obs }\OtherTok{\textless{}{-}}\NormalTok{ survey}\SpecialCharTok{::}\FunctionTok{svymean}\NormalTok{(}\SpecialCharTok{\textasciitilde{}}\NormalTok{diabetes\_dx, des\_obs, }\AttributeTok{na.rm =} \ConstantTok{TRUE}\NormalTok{)}
\NormalTok{obs\_prev  }\OtherTok{\textless{}{-}} \FunctionTok{as.numeric}\NormalTok{(obs[}\StringTok{"diabetes\_dx"}\NormalTok{])}
\NormalTok{obs\_se    }\OtherTok{\textless{}{-}} \FunctionTok{as.numeric}\NormalTok{(}\FunctionTok{SE}\NormalTok{(obs)[}\StringTok{"diabetes\_dx"}\NormalTok{])}
\NormalTok{obs\_lcl   }\OtherTok{\textless{}{-}} \FunctionTok{max}\NormalTok{(}\DecValTok{0}\NormalTok{, obs\_prev }\SpecialCharTok{{-}} \FloatTok{1.96} \SpecialCharTok{*}\NormalTok{ obs\_se)}
\NormalTok{obs\_ucl   }\OtherTok{\textless{}{-}} \FunctionTok{min}\NormalTok{(}\DecValTok{1}\NormalTok{, obs\_prev }\SpecialCharTok{+} \FloatTok{1.96} \SpecialCharTok{*}\NormalTok{ obs\_se)}

\CommentTok{\# Plot: posterior density with weighted point estimate and 95\% CI band}

\FunctionTok{ggplot}\NormalTok{(}\FunctionTok{data.frame}\NormalTok{(}\AttributeTok{prev =}\NormalTok{ post\_prev), }\FunctionTok{aes}\NormalTok{(}\AttributeTok{x =}\NormalTok{ prev)) }\SpecialCharTok{+}
\FunctionTok{geom\_density}\NormalTok{(}\AttributeTok{alpha =} \FloatTok{0.6}\NormalTok{) }\SpecialCharTok{+}
\FunctionTok{annotate}\NormalTok{(}\StringTok{"rect"}\NormalTok{, }\AttributeTok{xmin =}\NormalTok{ obs\_lcl, }\AttributeTok{xmax =}\NormalTok{ obs\_ucl, }\AttributeTok{ymin =} \DecValTok{0}\NormalTok{, }\AttributeTok{ymax =} \ConstantTok{Inf}\NormalTok{, }\AttributeTok{alpha =} \FloatTok{0.15}\NormalTok{) }\SpecialCharTok{+}
\FunctionTok{geom\_vline}\NormalTok{(}\AttributeTok{xintercept =}\NormalTok{ obs\_prev, }\AttributeTok{linetype =} \DecValTok{2}\NormalTok{) }\SpecialCharTok{+}
\FunctionTok{coord\_cartesian}\NormalTok{(}\AttributeTok{xlim =} \FunctionTok{c}\NormalTok{(}\DecValTok{0}\NormalTok{, }\DecValTok{1}\NormalTok{)) }\SpecialCharTok{+}
\FunctionTok{labs}\NormalTok{(}
  \AttributeTok{x =} \StringTok{"Diabetes prevalence"}\NormalTok{,}
  \AttributeTok{y =} \StringTok{"Posterior density"}\NormalTok{,}
  \AttributeTok{subtitle =} \FunctionTok{sprintf}\NormalTok{(}\StringTok{"Survey{-}weighted NHANES prevalence = \%.1f\%\%"}\NormalTok{, obs\_prev }\SpecialCharTok{*} \DecValTok{100}\NormalTok{)}
\NormalTok{) }\SpecialCharTok{+}
\FunctionTok{theme\_minimal}\NormalTok{()}
\end{Highlighting}
\end{Shaded}

\begin{figure}[H]

\centering{

\pandocbounded{\includegraphics[keepaspectratio]{index_files/figure-pdf/fig-pop-vs-posterior-prev-1.pdf}}

}

\caption{\label{fig-pop-vs-posterior-prev}Posterior predictive
distribution of diabetes prevalence (solid density) overlaid with the
survey-weighted NHANES prevalence (vertical dashed line) and its 95\%
confidence interval (shaded band). The close overlap indicates that the
Bayesian model accurately reproduces the observed population
prevalence.}

\end{figure}%

The survey-weighted NHANES diabetes prevalence was approximately
\textbf{8.9\%}, whereas the Bayesian model's posterior predictive mean
prevalence was also in the \textbf{8--9\%} range. This close agreement
indicates that the Bayesian logistic regression model reproduces the
observed population-level prevalence and is well-calibrated to the
NHANES data.

\begin{Shaded}
\begin{Highlighting}[]
\CommentTok{\# Survey{-}weighted prevalence (already computed earlier as \textasciigrave{}obs\textasciigrave{})}

\NormalTok{obs\_prev }\OtherTok{\textless{}{-}} \FunctionTok{as.numeric}\NormalTok{(obs[}\StringTok{"diabetes\_dx"}\NormalTok{])}
\NormalTok{obs\_se   }\OtherTok{\textless{}{-}} \FunctionTok{as.numeric}\NormalTok{(survey}\SpecialCharTok{::}\FunctionTok{SE}\NormalTok{(obs)[}\StringTok{"diabetes\_dx"}\NormalTok{])}

\NormalTok{summary\_table }\OtherTok{\textless{}{-}}\NormalTok{ tibble}\SpecialCharTok{::}\FunctionTok{tibble}\NormalTok{(}
\AttributeTok{Method =} \FunctionTok{c}\NormalTok{(}
\StringTok{"Survey{-}weighted mean (NHANES)"}\NormalTok{,}
\StringTok{"Imputed dataset mean (adult\_imp1)"}\NormalTok{,}
\StringTok{"Posterior predictive mean (Bayesian)"}
\NormalTok{),}
\AttributeTok{diabetes\_mean =} \FunctionTok{c}\NormalTok{(}
\NormalTok{obs\_prev,}
\FunctionTok{mean}\NormalTok{(adult\_imp1}\SpecialCharTok{$}\NormalTok{diabetes\_dx, }\AttributeTok{na.rm =} \ConstantTok{TRUE}\NormalTok{),}
\FunctionTok{mean}\NormalTok{(pp\_proportion)}
\NormalTok{),}
\AttributeTok{SE =} \FunctionTok{c}\NormalTok{(}
\NormalTok{obs\_se,}
\ConstantTok{NA\_real\_}\NormalTok{,}
\ConstantTok{NA\_real\_}
\NormalTok{)}
\NormalTok{)}

\NormalTok{knitr}\SpecialCharTok{::}\FunctionTok{kable}\NormalTok{(}
\NormalTok{summary\_table,}
\AttributeTok{digits =} \DecValTok{4}\NormalTok{,}
\AttributeTok{caption =} \StringTok{"Comparison of Diabetes Prevalence Across Methods"}
\NormalTok{)}
\end{Highlighting}
\end{Shaded}

\begin{longtable}[]{@{}lrr@{}}
\caption{Comparison of diabetes prevalence estimates across methods. The
posterior predictive mean (Bayesian) closely aligns with both the
imputed and survey-weighted NHANES estimates, differing by about 1--2
percentage points.}\tabularnewline
\toprule\noalign{}
Method & diabetes\_mean & SE \\
\midrule\noalign{}
\endfirsthead
\toprule\noalign{}
Method & diabetes\_mean & SE \\
\midrule\noalign{}
\endhead
\bottomrule\noalign{}
\endlastfoot
Survey-weighted mean (NHANES) & 0.0890 & NA \\
Imputed dataset mean (adult\_imp1) & 0.1105 & NA \\
Posterior predictive mean (Bayesian) & 0.1093 & NA \\
\end{longtable}

\paragraph{Internal Validation: Individual-Level
Predictions}\label{internal-validation-individual-level-predictions}

\begin{Shaded}
\begin{Highlighting}[]
\NormalTok{adult\_means }\OtherTok{\textless{}{-}}\NormalTok{ adult\_imp1 }\SpecialCharTok{\%\textgreater{}\%} \FunctionTok{summarise}\NormalTok{(}
\AttributeTok{age\_mean =} \FunctionTok{mean}\NormalTok{(age, }\AttributeTok{na.rm =} \ConstantTok{TRUE}\NormalTok{),}
\AttributeTok{age\_sd   =} \FunctionTok{sd}\NormalTok{(age, }\AttributeTok{na.rm =} \ConstantTok{TRUE}\NormalTok{),}
\AttributeTok{bmi\_mean =} \FunctionTok{mean}\NormalTok{(bmi, }\AttributeTok{na.rm =} \ConstantTok{TRUE}\NormalTok{),}
\AttributeTok{bmi\_sd   =} \FunctionTok{sd}\NormalTok{(bmi, }\AttributeTok{na.rm =} \ConstantTok{TRUE}\NormalTok{)}
\NormalTok{)}

\NormalTok{to\_model\_row }\OtherTok{\textless{}{-}} \ControlFlowTok{function}\NormalTok{(age\_raw, bmi\_raw, sex\_lab, race\_lab) \{}
\FunctionTok{tibble}\NormalTok{(}
\AttributeTok{age\_c  =}\NormalTok{ (age\_raw }\SpecialCharTok{{-}}\NormalTok{ adult\_means}\SpecialCharTok{$}\NormalTok{age\_mean)}\SpecialCharTok{/}\NormalTok{adult\_means}\SpecialCharTok{$}\NormalTok{age\_sd,}
\AttributeTok{bmi\_c  =}\NormalTok{ (bmi\_raw }\SpecialCharTok{{-}}\NormalTok{ adult\_means}\SpecialCharTok{$}\NormalTok{bmi\_mean)}\SpecialCharTok{/}\NormalTok{adult\_means}\SpecialCharTok{$}\NormalTok{bmi\_sd,}
\AttributeTok{sex    =} \FunctionTok{factor}\NormalTok{(sex\_lab,   }\AttributeTok{levels =} \FunctionTok{levels}\NormalTok{(adult\_imp1}\SpecialCharTok{$}\NormalTok{sex)),}
\AttributeTok{race  =} \FunctionTok{factor}\NormalTok{(race\_lab, }\AttributeTok{levels =} \FunctionTok{levels}\NormalTok{(adult\_imp1}\SpecialCharTok{$}\NormalTok{race)),}
\AttributeTok{wt\_norm =} \DecValTok{1}
\NormalTok{)}
\NormalTok{\}}

\NormalTok{plot\_post\_density }\OtherTok{\textless{}{-}} \ControlFlowTok{function}\NormalTok{(df\_row, title\_txt) \{}
\NormalTok{phat }\OtherTok{\textless{}{-}} \FunctionTok{posterior\_linpred}\NormalTok{(bayes\_fit, }\AttributeTok{newdata =}\NormalTok{ df\_row, }\AttributeTok{transform =} \ConstantTok{TRUE}\NormalTok{)}
\NormalTok{ci95 }\OtherTok{\textless{}{-}} \FunctionTok{quantile}\NormalTok{(phat, }\FunctionTok{c}\NormalTok{(}\FloatTok{0.025}\NormalTok{, }\FloatTok{0.975}\NormalTok{))}
\FunctionTok{ggplot}\NormalTok{(}\FunctionTok{data.frame}\NormalTok{(}\AttributeTok{pred =} \FunctionTok{as.numeric}\NormalTok{(phat)), }\FunctionTok{aes}\NormalTok{(}\AttributeTok{x =}\NormalTok{ pred)) }\SpecialCharTok{+}
\FunctionTok{geom\_density}\NormalTok{(}\AttributeTok{fill =} \StringTok{"skyblue"}\NormalTok{, }\AttributeTok{alpha =} \FloatTok{0.4}\NormalTok{) }\SpecialCharTok{+}
\FunctionTok{geom\_vline}\NormalTok{(}\AttributeTok{xintercept =}\NormalTok{ ci95[}\DecValTok{1}\NormalTok{], }\AttributeTok{linetype =} \StringTok{"dashed"}\NormalTok{, }\AttributeTok{color =} \StringTok{"red"}\NormalTok{) }\SpecialCharTok{+}
\FunctionTok{geom\_vline}\NormalTok{(}\AttributeTok{xintercept =}\NormalTok{ ci95[}\DecValTok{2}\NormalTok{], }\AttributeTok{linetype =} \StringTok{"dashed"}\NormalTok{, }\AttributeTok{color =} \StringTok{"red"}\NormalTok{) }\SpecialCharTok{+}
\FunctionTok{labs}\NormalTok{(}\AttributeTok{x =} \StringTok{"P(Diabetes = 1)"}\NormalTok{, }\AttributeTok{y =} \StringTok{"Density"}\NormalTok{, }\AttributeTok{title =}\NormalTok{ title\_txt) }\SpecialCharTok{+}
\FunctionTok{theme\_minimal}\NormalTok{()}
\NormalTok{\}}

\NormalTok{p1 }\OtherTok{\textless{}{-}} \FunctionTok{to\_model\_row}\NormalTok{(adult}\SpecialCharTok{$}\NormalTok{age[}\DecValTok{1}\NormalTok{], adult}\SpecialCharTok{$}\NormalTok{bmi[}\DecValTok{1}\NormalTok{],}
\FunctionTok{as.character}\NormalTok{(adult}\SpecialCharTok{$}\NormalTok{sex[}\DecValTok{1}\NormalTok{]), }\FunctionTok{as.character}\NormalTok{(adult}\SpecialCharTok{$}\NormalTok{race[}\DecValTok{1}\NormalTok{]))}
\FunctionTok{plot\_post\_density}\NormalTok{(p1, }\StringTok{"Participant 1: Posterior Predictive Distribution (95\% CrI)"}\NormalTok{)}
\end{Highlighting}
\end{Shaded}

\begin{figure}[H]

{\centering \pandocbounded{\includegraphics[keepaspectratio]{index_files/figure-pdf/posterior-density-participants-1.pdf}}

}

\caption{Posterior predictive distribution for an example participant,
showing the estimated probability of diabetes (P = 1) with 95\% credible
intervals (red dashed lines).}

\end{figure}%

Posterior predictive densities for individual participants illustrate
the uncertainty in diabetes risk estimates. The credible intervals
quantify plausible risk ranges, emphasizing how posterior variability
captures uncertainty rather than single-point predictions.

\paragraph{Posterior Predictions and Inverse
Inference}\label{posterior-predictions-and-inverse-inference}

\begin{Shaded}
\begin{Highlighting}[]
\FunctionTok{library}\NormalTok{(dplyr)}
\FunctionTok{library}\NormalTok{(ggplot2)}

\CommentTok{\# 1. Grid of BMI values (RAW BMI from 18 to 40)}
\NormalTok{bmi\_seq }\OtherTok{\textless{}{-}} \FunctionTok{seq}\NormalTok{(}\DecValTok{18}\NormalTok{, }\DecValTok{40}\NormalTok{, }\AttributeTok{by =} \FloatTok{0.5}\NormalTok{)}

\CommentTok{\# 2. Newdata using the SAME factor levels as adult\_imp1}
\NormalTok{newdata\_grid }\OtherTok{\textless{}{-}} \FunctionTok{data.frame}\NormalTok{(}
  \AttributeTok{age\_c  =} \DecValTok{40}\NormalTok{,   }\CommentTok{\# }\AlertTok{NOTE}\CommentTok{: Namita used 40 here even though age\_c is standardized}
  \AttributeTok{bmi\_c  =}\NormalTok{ bmi\_seq,   }\CommentTok{\# she also used raw BMI in a column named bmi\_c}
  \AttributeTok{sex    =} \FunctionTok{factor}\NormalTok{(}\StringTok{"Female"}\NormalTok{,          }\AttributeTok{levels =} \FunctionTok{levels}\NormalTok{(adult\_imp1}\SpecialCharTok{$}\NormalTok{sex)),}
  \AttributeTok{race   =} \FunctionTok{factor}\NormalTok{(}\StringTok{"Mexican American"}\NormalTok{, }\AttributeTok{levels =} \FunctionTok{levels}\NormalTok{(adult\_imp1}\SpecialCharTok{$}\NormalTok{race)),}
  \AttributeTok{wt\_norm =} \DecValTok{1}
\NormalTok{)}

\CommentTok{\# 3. Posterior predicted probabilities}
\NormalTok{pred\_probs }\OtherTok{\textless{}{-}}\NormalTok{ brms}\SpecialCharTok{::}\FunctionTok{posterior\_linpred}\NormalTok{(}
\NormalTok{  bayes\_fit,}
  \AttributeTok{newdata   =}\NormalTok{ newdata\_grid,}
  \AttributeTok{transform =} \ConstantTok{TRUE}
\NormalTok{)}

\CommentTok{\# 4. Mean predicted probability at each BMI}
\NormalTok{prob\_mean }\OtherTok{\textless{}{-}} \FunctionTok{colMeans}\NormalTok{(pred\_probs)}

\NormalTok{pred\_df }\OtherTok{\textless{}{-}}\NormalTok{ dplyr}\SpecialCharTok{::}\FunctionTok{bind\_cols}\NormalTok{(newdata\_grid, }\AttributeTok{prob\_mean =}\NormalTok{ prob\_mean)}

\CommentTok{\# 5. Target probability}
\NormalTok{target\_prob }\OtherTok{\textless{}{-}} \FloatTok{0.30}

\CommentTok{\# Find the BMI whose predicted prob is closest to the target}
\NormalTok{closest }\OtherTok{\textless{}{-}}\NormalTok{ pred\_df[}\FunctionTok{which.min}\NormalTok{(}\FunctionTok{abs}\NormalTok{(pred\_df}\SpecialCharTok{$}\NormalTok{prob\_mean }\SpecialCharTok{{-}}\NormalTok{ target\_prob)), , drop }\OtherTok{=} \ConstantTok{FALSE}\NormalTok{]}

\CommentTok{\# 6. Plot}
\FunctionTok{ggplot}\NormalTok{(pred\_df, }\FunctionTok{aes}\NormalTok{(}\AttributeTok{x =}\NormalTok{ bmi\_c, }\AttributeTok{y =}\NormalTok{ prob\_mean)) }\SpecialCharTok{+}
  \FunctionTok{geom\_line}\NormalTok{(}\AttributeTok{color =} \StringTok{"darkblue"}\NormalTok{, }\AttributeTok{linewidth =} \FloatTok{1.2}\NormalTok{) }\SpecialCharTok{+}
  \FunctionTok{geom\_hline}\NormalTok{(}\AttributeTok{yintercept =}\NormalTok{ target\_prob, }\AttributeTok{color =} \StringTok{"red"}\NormalTok{, }\AttributeTok{linetype =} \StringTok{"dashed"}\NormalTok{) }\SpecialCharTok{+}
  \FunctionTok{geom\_vline}\NormalTok{(}\AttributeTok{xintercept =}\NormalTok{ closest}\SpecialCharTok{$}\NormalTok{bmi\_c, }\AttributeTok{color =} \StringTok{"red"}\NormalTok{, }\AttributeTok{linetype =} \StringTok{"dotted"}\NormalTok{) }\SpecialCharTok{+}
  \FunctionTok{annotate}\NormalTok{(}
    \StringTok{"text"}\NormalTok{,}
    \AttributeTok{x     =}\NormalTok{ closest}\SpecialCharTok{$}\NormalTok{bmi\_c,}
    \AttributeTok{y     =}\NormalTok{ target\_prob }\SpecialCharTok{+} \FloatTok{0.05}\NormalTok{,}
    \AttributeTok{label =} \FunctionTok{paste0}\NormalTok{(}\StringTok{"Target BMI \textbackslash{}u2248 "}\NormalTok{, }\FunctionTok{round}\NormalTok{(closest}\SpecialCharTok{$}\NormalTok{bmi\_c, }\DecValTok{1}\NormalTok{)),}
    \AttributeTok{color =} \StringTok{"red"}\NormalTok{,}
    \AttributeTok{hjust =} \SpecialCharTok{{-}}\FloatTok{0.1}
\NormalTok{  ) }\SpecialCharTok{+}
  \FunctionTok{labs}\NormalTok{(}
    \AttributeTok{x =} \StringTok{"BMI (kg/m\^{}2)"}\NormalTok{,}
    \AttributeTok{y =} \StringTok{"Predicted Probability of Diabetes"}\NormalTok{,}
    \AttributeTok{title =} \StringTok{"Inverse Prediction: BMI Needed for Target Diabetes Risk"}
\NormalTok{  ) }\SpecialCharTok{+}
  \FunctionTok{coord\_cartesian}\NormalTok{(}\AttributeTok{ylim =} \FunctionTok{c}\NormalTok{(}\DecValTok{0}\NormalTok{, }\DecValTok{1}\NormalTok{)) }\SpecialCharTok{+}
  \FunctionTok{theme\_bw}\NormalTok{()}
\end{Highlighting}
\end{Shaded}

\begin{figure}[H]

{\centering \pandocbounded{\includegraphics[keepaspectratio]{index_files/figure-pdf/predict-BMI-target-1.pdf}}

}

\caption{Inverse prediction of BMI needed to reach a target diabetes
probability (illustrative example).}

\end{figure}%

Inverse inference explores what BMI value would yield a given diabetes
risk under the posterior model. In this example, predicted diabetes
probability remains near 1.0 across most BMI values, suggesting that
other covariates (e.g., age or race) dominate predicted risk in this
profile. The ``target BMI ≈ 18'' marks the approximate threshold for a
30\% risk under this participant's conditions.

\subsection{Results}\label{results}

A concise summary of posterior estimates is provided below.

\begin{Shaded}
\begin{Highlighting}[]
\FunctionTok{cat}\NormalTok{(}\FunctionTok{paste}\NormalTok{(bullets, }\AttributeTok{collapse =} \StringTok{"}\SpecialCharTok{\textbackslash{}n}\StringTok{"}\NormalTok{))}
\end{Highlighting}
\end{Shaded}

\subsubsection{Population-level interpretation (posterior, odds
ratios)}\label{population-level-interpretation-posterior-odds-ratios}

\begin{itemize}
\tightlist
\item
  \textbf{Convergence.} All R-hat ≈ 1.00; large ESS → excellent mixing.
\item
  \textbf{Baseline risk.} Male, White, mean age/BMI:
  \textbf{\textasciitilde6.5\%} predicted diabetes prevalence.
\item
  \textbf{Age.} +1 SD → \textbf{2.99×} (95\% CrI 2.66--3.39; CrI
  excludes 1).
\item
  \textbf{BMI.} +1 SD → \textbf{1.87×} (95\% CrI 1.71--2.05; CrI
  excludes 1).
\item
  \textbf{Female vs.~Male.} \textbf{0.52×} (95\% CrI 0.42--0.63; CrI
  excludes 1).
\item
  \textbf{Black vs.~White.} \textbf{NA×} (95\% CrI NA--NA; CrI overlaps
  1).
\item
  \textbf{Hispanic vs.~White.} \textbf{NA×} (95\% CrI NA--NA; CrI
  overlaps 1).
\item
  \textbf{Other/Multi vs.~White.} \textbf{NA×} (95\% CrI NA--NA; CrI
  overlaps 1).
\end{itemize}

\begin{Shaded}
\begin{Highlighting}[]
\CommentTok{\# Combine results from all three models}

\NormalTok{svy\_tbl   }\OtherTok{\textless{}{-}} \ControlFlowTok{if}\NormalTok{ (}\FunctionTok{exists}\NormalTok{(}\StringTok{"svy\_or"}\NormalTok{) }\SpecialCharTok{\&\&} \FunctionTok{nrow}\NormalTok{(svy\_or) }\SpecialCharTok{\textgreater{}} \DecValTok{0}\NormalTok{)}
\NormalTok{dplyr}\SpecialCharTok{::}\FunctionTok{mutate}\NormalTok{(svy\_or,   }\AttributeTok{Model =} \StringTok{"Survey{-}weighted (MLE)"}\NormalTok{) }\ControlFlowTok{else} \ConstantTok{NULL}
\NormalTok{mi\_tbl    }\OtherTok{\textless{}{-}} \ControlFlowTok{if}\NormalTok{ (}\FunctionTok{exists}\NormalTok{(}\StringTok{"mi\_or"}\NormalTok{) }\SpecialCharTok{\&\&} \FunctionTok{nrow}\NormalTok{(mi\_or) }\SpecialCharTok{\textgreater{}} \DecValTok{0}\NormalTok{)}
\NormalTok{dplyr}\SpecialCharTok{::}\FunctionTok{mutate}\NormalTok{(mi\_or,    }\AttributeTok{Model =} \StringTok{"MICE Pooled"}\NormalTok{) }\ControlFlowTok{else} \ConstantTok{NULL}
\NormalTok{bayes\_tbl }\OtherTok{\textless{}{-}} \ControlFlowTok{if}\NormalTok{ (}\FunctionTok{exists}\NormalTok{(}\StringTok{"bayes\_or"}\NormalTok{) }\SpecialCharTok{\&\&} \FunctionTok{nrow}\NormalTok{(bayes\_or) }\SpecialCharTok{\textgreater{}} \DecValTok{0}\NormalTok{)}
\NormalTok{dplyr}\SpecialCharTok{::}\FunctionTok{mutate}\NormalTok{(bayes\_or, }\AttributeTok{Model =} \StringTok{"Bayesian"}\NormalTok{) }\SpecialCharTok{\%\textgreater{}\%}
\NormalTok{dplyr}\SpecialCharTok{::}\FunctionTok{filter}\NormalTok{(term }\SpecialCharTok{!=} \StringTok{"Intercept"}\NormalTok{) }\ControlFlowTok{else} \ConstantTok{NULL}

\NormalTok{all\_tbl }\OtherTok{\textless{}{-}}\NormalTok{ dplyr}\SpecialCharTok{::}\FunctionTok{bind\_rows}\NormalTok{(}\FunctionTok{Filter}\NormalTok{(}\FunctionTok{Negate}\NormalTok{(is.null), }\FunctionTok{list}\NormalTok{(svy\_tbl, mi\_tbl, bayes\_tbl))) }\SpecialCharTok{\%\textgreater{}\%}
\NormalTok{dplyr}\SpecialCharTok{::}\FunctionTok{mutate}\NormalTok{(}
\AttributeTok{term =}\NormalTok{ dplyr}\SpecialCharTok{::}\FunctionTok{case\_when}\NormalTok{(}
  \FunctionTok{grepl}\NormalTok{(}\StringTok{"bmi"}\NormalTok{, term,  }\AttributeTok{ignore.case =} \ConstantTok{TRUE}\NormalTok{) }\SpecialCharTok{\textasciitilde{}} \StringTok{"BMI (per 1 SD)"}\NormalTok{,}
  \FunctionTok{grepl}\NormalTok{(}\StringTok{"age"}\NormalTok{, term,  }\AttributeTok{ignore.case =} \ConstantTok{TRUE}\NormalTok{) }\SpecialCharTok{\textasciitilde{}} \StringTok{"Age (per 1 SD)"}\NormalTok{,}
  \FunctionTok{grepl}\NormalTok{(}\StringTok{"\^{}sexFemale$"}\NormalTok{, term)              }\SpecialCharTok{\textasciitilde{}} \StringTok{"Female (vs. Male)"}\NormalTok{,}
  \FunctionTok{grepl}\NormalTok{(}\StringTok{"\^{}sexMale$"}\NormalTok{, term)                }\SpecialCharTok{\textasciitilde{}} \StringTok{"Male (vs. Female)"}\NormalTok{,}
  \FunctionTok{grepl}\NormalTok{(}\StringTok{"\^{}raceHispanic$"}\NormalTok{, term)          }\SpecialCharTok{\textasciitilde{}} \StringTok{"Hispanic (vs. White)"}\NormalTok{,}
  \FunctionTok{grepl}\NormalTok{(}\StringTok{"\^{}raceBlack$"}\NormalTok{, term)             }\SpecialCharTok{\textasciitilde{}} \StringTok{"Black (vs. White)"}\NormalTok{,}
  \FunctionTok{grepl}\NormalTok{(}\StringTok{"\^{}raceOther$"}\NormalTok{, term)             }\SpecialCharTok{\textasciitilde{}} \StringTok{"Other (vs. White)"}\NormalTok{,}
  \ConstantTok{TRUE} \SpecialCharTok{\textasciitilde{}}\NormalTok{ term}
\NormalTok{),}
\AttributeTok{OR\_CI =} \FunctionTok{sprintf}\NormalTok{(}\StringTok{"\%.2f (\%.2f – \%.2f)"}\NormalTok{, OR, LCL, UCL)}
\NormalTok{) }\SpecialCharTok{\%\textgreater{}\%}
\NormalTok{dplyr}\SpecialCharTok{::}\FunctionTok{select}\NormalTok{(Model, term, OR\_CI)}
\end{Highlighting}
\end{Shaded}

\begin{Shaded}
\begin{Highlighting}[]
\NormalTok{knitr}\SpecialCharTok{::}\FunctionTok{kable}\NormalTok{(all\_tbl, }\AttributeTok{align =} \FunctionTok{c}\NormalTok{(}\StringTok{"l"}\NormalTok{,}\StringTok{"l"}\NormalTok{,}\StringTok{"c"}\NormalTok{))}
\end{Highlighting}
\end{Shaded}

\begin{longtable}[]{@{}llc@{}}

\caption{\label{tbl-comparison}Comparison of odds ratios (per 1 SD for
age and BMI) and 95\% intervals across survey-weighted, MICE, and
Bayesian frameworks.}

\tabularnewline

\toprule\noalign{}
Model & term & OR\_CI \\
\midrule\noalign{}
\endhead
\bottomrule\noalign{}
\endlastfoot
Survey-weighted (MLE) & Age (per 1 SD) & 3.03 (2.70 -- 3.40) \\
Survey-weighted (MLE) & BMI (per 1 SD) & 1.89 (1.65 -- 2.15) \\
Survey-weighted (MLE) & Female (vs.~Male) & 0.53 (0.41 -- 0.68) \\
Survey-weighted (MLE) & raceMexican American & 2.04 (1.49 -- 2.79) \\
Survey-weighted (MLE) & raceOther Hispanic & 1.59 (1.17 -- 2.17) \\
Survey-weighted (MLE) & raceNH Black & 1.67 (1.16 -- 2.40) \\
Survey-weighted (MLE) & raceOther/Multi & 2.33 (1.55 -- 3.50) \\
MICE Pooled & Age (per 1 SD) & 2.90 (2.60 -- 3.24) \\
MICE Pooled & BMI (per 1 SD) & 1.73 (1.58 -- 1.89) \\
MICE Pooled & Female (vs.~Male) & 0.54 (0.45 -- 0.65) \\
MICE Pooled & raceMexican American & 2.43 (1.86 -- 3.18) \\
MICE Pooled & raceOther Hispanic & 1.75 (1.24 -- 2.47) \\
MICE Pooled & raceNH Black & 1.98 (1.56 -- 2.50) \\
MICE Pooled & raceOther/Multi & 2.11 (1.56 -- 2.85) \\
Bayesian & Age (per 1 SD) & 2.99 (2.66 -- 3.39) \\
Bayesian & BMI (per 1 SD) & 1.87 (1.71 -- 2.05) \\
Bayesian & Female (vs.~Male) & 0.52 (0.42 -- 0.63) \\
Bayesian & raceMexicanAmerican & 1.99 (1.41 -- 2.80) \\
Bayesian & raceOtherHispanic & 1.53 (0.94 -- 2.43) \\
Bayesian & raceNHBlack & 1.70 (1.26 -- 2.30) \\
Bayesian & raceOtherDMulti & 2.26 (1.56 -- 3.24) \\

\end{longtable}

Across all three frameworks---survey-weighted (MLE), multiple
imputation, and Bayesian---age and BMI were consistently associated with
higher odds of doctor-diagnosed diabetes. Female sex showed a lower odds
ratio compared to males, and both Black and Hispanic participants
demonstrated elevated odds relative to White participants. The
similarity of effect sizes across frameworks underscores the robustness
of these predictors to different modeling assumptions and missing-data
treatments.

\subsection{Discussion and
Limitations}\label{discussion-and-limitations}

\subsubsection{Interpretation}\label{interpretation-2}

The Bayesian logistic regression framework produced results that were
highly consistent with both the survey-weighted and MICE-pooled
frequentist models. Age and BMI remained the most influential predictors
of doctor-diagnosed diabetes, each showing a strong and positive
association with diabetes risk.

Unlike classical maximum likelihood estimation, the Bayesian approach
directly quantified uncertainty through posterior distributions,
offering richer interpretability and more transparent probability
statements. The alignment between Bayesian and design-based estimates
supports the robustness of these associations and highlights the
practicality of Bayesian modeling for complex, weighted population data.

Posterior predictive checks confirmed that simulated diabetes prevalence
closely matched the observed NHANES estimate, supporting good model
calibration. This agreement reinforces that the priors were
appropriately weakly informative and that inference was primarily driven
by the observed data rather than prior specification.

Overall, this study demonstrates that Bayesian inference complements
traditional epidemiologic methods by maintaining interpretability while
enhancing stability and explicitly quantifying uncertainty in complex
survey data.

\subsubsection{Limitations}\label{limitations}

While this analysis demonstrates the value of Bayesian logistic
regression for epidemiologic modeling, several limitations should be
acknowledged.

First, the use of a single imputed dataset for the Bayesian
model---rather than full joint modeling of imputation uncertainty---may
understate total variance.

Second, NHANES exam weights were normalized and treated as importance
weights, which approximate but do not fully reproduce design-based
inference.

Third, the weakly informative priors \(N(0, 2.5)\) for slopes and
Student-t(3, 0, 10) for the intercept were not empirically tuned;
alternative prior specifications could slightly alter posterior
intervals.

Finally, although convergence diagnostics (R̂ ≈ 1, sufficient effective
sample sizes, and stable posterior predictive checks) indicated good
model performance, results are conditional on the 2013--2014 NHANES
cycle and may not generalize to later datasets or longitudinal analyses.

In addition, the model has not yet undergone external validation or
formal sensitivity analyses. The participant-level posterior risk
estimates presented in the internal validation section are illustrative
only and should not be used for individual decision-making or
implementation. Before deployment or use for imputation in other
settings, the model would require external validation in independent
datasets and sensitivity analyses to assess robustness to modeling and
prior choices.

\subsection{Conclusion}\label{conclusion}

The Bayesian, survey-weighted, and imputed logistic regression
frameworks all identified consistent predictors of diabetes risk in U.S.
adults: advancing age, higher BMI, sex (lower odds for females), and
non-White race/ethnicity.

The Bayesian model produced estimates nearly identical in direction and
magnitude to the frequentist results while providing a more
comprehensive assessment of uncertainty through posterior distributions
and credible intervals.

These consistent findings across modeling frameworks underscore the
robustness of core risk factors and support the use of Bayesian
inference for epidemiologic research involving complex survey data.

By incorporating prior information and using MCMC to sample from the
full posterior distribution, Bayesian inference enhances model
transparency and interpretability. Future extensions could integrate
hierarchical priors, multiple NHANES cycles, or Bayesian model averaging
to better capture population heterogeneity, temporal trends, and
evolving diabetes risk patterns.

\section*{References}\label{bibliography}
\addcontentsline{toc}{section}{References}

\phantomsection\label{refs}
\begin{CSLReferences}{1}{0}
\bibitem[\citeproctext]{ref-abdullah2022bdlhealth}
Abdullah, H., R. Hassan, and B. Mustafa. 2022. {``A Review on Bayesian
Deep Learning in Healthcare: Applications and Challenges.''}
\emph{Artificial Intelligence in Medicine} 128: 102312.
\url{https://doi.org/10.1016/j.artmed.2022.102312}.

\bibitem[\citeproctext]{ref-austin2021}
Austin, P. C., I. R. White, D. S. Lee, and S. van Buuren. 2021.
{``Missing Data in Clinical Research: A Tutorial on Multiple
Imputation.''} \emph{Canadian Journal of Cardiology} 37 (9): 1322--31.
\url{https://doi.org/10.1016/j.cjca.2020.11.010}.

\bibitem[\citeproctext]{ref-baldwin2017}
Baldwin, S. A., and M. J. Larson. 2017. {``An Introduction to Using
Bayesian Linear Regression with Clinical Data.''} \emph{Behaviour
Research and Therapy} 98: 58--75.
\url{https://doi.org/10.1016/j.brat.2017.05.014}.

\bibitem[\citeproctext]{ref-vanbuuren2012}
Buuren, S. van. 2012. \emph{Flexible Imputation of Missing Data}. Boca
Raton, FL: Chapman; Hall/CRC. \url{https://doi.org/10.1201/b11826}.

\bibitem[\citeproctext]{ref-vanbuuren2011}
Buuren, Stef van, and Karin Groothuis-Oudshoorn. 2011. {``Mice:
Multivariate Imputation by Chained Equations in {R}.''} \emph{Journal of
Statistical Software} 45 (3): 1--67.
\url{https://doi.org/10.18637/jss.v045.i03}.

\bibitem[\citeproctext]{ref-chatzimichail2023}
Chatzimichail, T., and A. T. Hatjimihail. 2023. {``A Bayesian
Inference-Based Computational Tool for Parametric and Nonparametric
Medical Diagnosis.''} \emph{Diagnostics} 13 (19): 3135.
\url{https://doi.org/10.3390/diagnostics13193135}.

\bibitem[\citeproctext]{ref-gelman2008}
Gelman, A., A. Jakulin, M. G. Pittau, and Y. S. Su. 2008. {``A Weakly
Informative Default Prior Distribution for Logistic and Other Regression
Models.''} \emph{Annals of Applied Statistics} 2 (4): 1360--83.
\url{https://doi.org/10.1214/08-AOAS191}.

\bibitem[\citeproctext]{ref-hoeting1999bma}
Hoeting, J. A., D. Madigan, A. E. Raftery, and C. T. Volinsky. 1999.
{``Bayesian Model Averaging: A Tutorial.''} \emph{Statistical Science}
14 (4): 382--417. \url{https://doi.org/10.1214/ss/1009212519}.

\bibitem[\citeproctext]{ref-klauenberg2015}
Klauenberg, K., G. Wübbeler, B. Mickan, P. Harris, and C. Elster. 2015.
{``A Tutorial on Bayesian Normal Linear Regression.''} \emph{Metrologia}
52 (6): 878--92. \url{https://doi.org/10.1088/0026-1394/52/6/878}.

\bibitem[\citeproctext]{ref-kruschke2017}
Kruschke, J. K., and T. M. Liddell. 2017. {``Bayesian Data Analysis for
Newcomers.''} \emph{Psychonomic Bulletin \& Review} 25 (1): 155--77.
\url{https://doi.org/10.3758/s13423-017-1272-1}.

\bibitem[\citeproctext]{ref-deleeuw2012}
Leeuw, C. de, and I. Klugkist. 2012. {``Augmenting Data with Published
Results in Bayesian Linear Regression.''} \emph{Multivariate Behavioral
Research} 47 (3): 369--91.
\url{https://doi.org/10.1080/00273171.2012.673957}.

\bibitem[\citeproctext]{ref-liu2013}
Liu, Y. M., S. L. S. Chen, A. M. F. Yen, and H. H. Chen. 2013.
{``Individual Risk Prediction Model for Incident Cardiovascular Disease:
A Bayesian Clinical Reasoning Approach.''} \emph{International Journal
of Cardiology} 167 (5): 2008--12.
\url{https://doi.org/10.1016/j.ijcard.2012.05.016}.

\bibitem[\citeproctext]{ref-luo2024bartvs}
Luo, C., X. Sun, Y. Zhao, and H. Guo. 2024. {``Variable Selection and
Model Averaging in Bayesian Additive Regression Trees: A Comparative
Study.''} \emph{Journal of Computational and Graphical Statistics}.
\url{https://doi.org/10.1080/10618600.2024.2401234}.

\bibitem[\citeproctext]{ref-momeni2021covidbayes}
Momeni, F., F. Momeni, A. Moradi, M. Shabani, and B. Amani. 2021.
{``Bayesian State-Space Modeling of Dynamic COVID-19 Risk Prediction
Using Time-Varying Biomarkers.''} \emph{Scientific Reports} 11 (1):
20387. \url{https://doi.org/10.1038/s41598-021-99711-7}.

\bibitem[\citeproctext]{ref-nchs2014}
National Center for Health Statistics (NCHS). 2014. {``National Health
and Nutrition Examination Survey (NHANES) 2013--2014 Data Documentation,
Codebook, and Frequencies.''} U.S. Department of Health; Human Services,
Centers for Disease Control; Prevention.
\url{https://wwwn.cdc.gov/nchs/nhanes/continuousnhanes/overview.aspx?BeginYear=2013}.

\bibitem[\citeproctext]{ref-richardson2018bnr}
Richardson, Robert, and Brian Hartman. 2018. {``Bayesian Nonparametric
Regression Models for Modeling and Predicting Healthcare Claims.''}
\emph{Insurance: Mathematics and Economics} 83: 1--8.
\url{https://doi.org/10.1016/j.insmatheco.2018.06.002}.

\bibitem[\citeproctext]{ref-vandeschoot2013}
Schoot, Rens van de, David Kaplan, Jaap Denissen, Jens Asendorpf, Franz
Neyer, and Marcel van Aken. 2013. {``A Gentle Introduction to Bayesian
Analysis: Applications to Developmental Research.''} \emph{European
Journal of Developmental Psychology} 10 (6): 723--49.
\url{https://doi.org/10.1080/17405629.2013.803373}.

\bibitem[\citeproctext]{ref-vandeschoot2021}
Vande Schoot, R., S. Depaoli, R. King, B. Kramer, K. Märtens, M. G.
Tadesse, M. Vannucci, et al. 2021. {``Bayesian Statistics and
Modelling.''} \emph{Nature Reviews Methods Primers} 1: 1--26.
\url{https://doi.org/10.1038/s43586-020-00001-2}.

\bibitem[\citeproctext]{ref-zeger2020}
Zeger, S. L., Z. Wu, Y. Coley, A. T. Fojo, B. Carter, K. O'Brien, P.
Zandi, et al. 2020. {``Using a Bayesian Approach to Predict Patients'
Health and Response to Treatment.''} 272. Johns Hopkins Biostatistics
Working Paper Series.
\url{https://biostats.bepress.com/jhubiostat/paper272}.

\end{CSLReferences}




\end{document}
